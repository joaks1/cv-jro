\documentclass[10pt]{article}
%\usepackage{anysize}
%\papersize{11in}{8.5in}
%\marginsize{1in}{1in}{.5in}{.5in}
\textwidth = 6.5 in
\textheight = 9 in
\oddsidemargin = 0.0 in
\evensidemargin = 0.0 in
\topmargin = -0.5 in
\headheight = 0.35 in
\headsep = 0.15 in
\topskip = 0 in
\footskip = 0.5 in
\pagenumbering{arabic}
\usepackage{setspace}
\usepackage[usenames]{color}
\usepackage[fleqn]{amsmath}
\usepackage{graphicx}
\usepackage{url}
\usepackage{verbatim}
\usepackage{indentfirst}
\usepackage{booktabs}
\usepackage{multirow}
\usepackage[table]{xcolor}
\usepackage{ragged2e}
\usepackage{xspace}
\usepackage{parskip}
\usepackage{tabulary}
\usepackage{longtable}
\usepackage[normalem]{ulem}
\usepackage{hyperref}
\hypersetup{pdfborder={0 0 0}, colorlinks=true, urlcolor=blue, linkcolor=black}
\usepackage{titlesec}
\usepackage{lastpage}
\usepackage{fancyhdr}
\usepackage{enumitem}
\usepackage{ifthen}

\setcounter{secnumdepth}{4}
\setcounter{tocdepth}{4}

%% Bibliography Formating %%%%%%%%%%%%%%%%%%%%%%%%%%%%%%%%%%%%%%%%%%%
\usepackage[bibstyle=bib/joaks-au-cv,backend=biber]{biblatex}

% Count total number of entries in each refsection
% \AtDataInput{%
%   \csnumgdef{entrycount:\therefsection}{%
%     \csuse{entrycount:\therefsection}+1}}

% Print the labelnumber as the total number of entries in the
% current refsection, minus the actual labelnumber, plus one
% \DeclareFieldFormat{labelnumber}{\mkbibdesc{#1}}
% \newrobustcmd*{\mkbibdesc}[1]{%
%   \number\numexpr\csuse{entrycount:\therefsection}+1-#1\relax}

\bibliography{../bib/jro}

\setlength{\bibitemsep}{0.5em}
\setlength{\bibhang}{5mm}

%% Format headers and footers %%%%%%%%%%%%%%%%%%%%
\pagestyle{fancy}
\lhead{\sffamily \emph{\docTitle} \\ Jamie R.\ Oaks}
\rhead{\sffamily \today}
\cfoot{\sffamily\footnotesize Page \thepage\ of \pageref{LastPage}}
\renewcommand{\headrulewidth}{0.4pt}
\renewcommand{\footrulewidth}{0pt}

%% Format section titles %%%%%%%%%%%%%%%%%%%%%%%%%
\titleformat{\section}[hang]
    {\normalfont\Large\bfseries}
    {\thesection}{1em}{}[]
\titlespacing*{\section}
    {0ex}{1.0ex plus .1ex minus .1ex}{0ex}
    % {0ex}{0ex}{0ex}

\titleformat{\subsection}[hang]
    {\normalfont\large\bfseries}
    {\thesubsection}{1em}{}[]
\titlespacing*{\subsection}
    {0ex}{1.0ex plus .1ex minus .1ex}{0ex}
    % {0ex}{0ex}{0ex}

\titleformat{\subsubsection}[hang]
    {\normalfont\normalsize\bfseries}
    {\thesubsubsection}{1em}{}[]
\titlespacing*{\subsubsection}
    {0ex}{1.0ex plus .1ex minus .1ex}{0ex}

\titleformat{\paragraph}[hang]
    {\normalsize\slshape\bfseries}
    {\theparagraph}{1em}{}[]
\titlespacing{\paragraph}
    {0ex}{0.5ex plus .1ex minus .1ex}{0.5ex}

\titleformat{\subparagraph}[runin]
    {\normalsize\slshape\bfseries}
    {}{}{}[---]
\titlespacing{\subparagraph}
    {\parindent}{0ex}{0.5ex}

\newenvironment{mytitle}
 {\parskip=0pt\par\nopagebreak\centering\large\slshape\bfseries}
 {\par\noindent\ignorespacesafterend}
\newenvironment{tightCenter}
 {\parskip=0pt\par\nopagebreak\centering}
 {\par\noindent\ignorespacesafterend}

%% Format list environments %%%%%%%%%%%%%%%%%%%%%%%%
\renewcommand{\labelenumii}{\arabic{enumi}.\arabic{enumii}}
\renewcommand{\labelitemi}{$\circ$}

\newenvironment{myEnumerate}{
  \begin{enumerate}
    \setlength{\itemsep}{0.25em}
    \setlength{\parskip}{0pt}
    \setlength{\parsep}{0.5em}}
  {\end{enumerate}}

\newenvironment{myItemize}{
  \begin{itemize}
    \setlength{\leftskip}{-4mm}
    \setlength{\itemsep}{0.25em}
    \setlength{\parskip}{0pt}
    \setlength{\parsep}{0.5em}}
  {\end{itemize}}

\newenvironment{tightItemize}{%
\begin{itemize}[noitemsep, topsep=0pt, parsep=0pt, partopsep=0pt]}
{\end{itemize}}

\newenvironment{veryTightItemize}{%
\begin{itemize}[noitemsep, topsep=0pt, parsep=0pt, partopsep=0pt, leftmargin=*]}
{\end{itemize}}

\newenvironment{tightEnumerate}{%
\begin{enumerate}[noitemsep, topsep=0pt, parsep=0pt, partopsep=0pt]}
{\end{enumerate}}

\newenvironment{veryTightEnumerate}{%
\begin{enumerate}[noitemsep, topsep=0pt, parsep=0pt, partopsep=0pt, leftmargin=*]}
{\end{enumerate}}

%% Basic formatting and spacing %%%%%%%%%%%%%%%%%%%%%
\setlength{\parindent}{0em}
\setlength{\parskip}{0.5em}

%% My functions %%%%%%%%%%%%%%%%%%%%%%%%%%%%%
\newcommand{\ignore}[1]{}
\newcommand{\addTail}[1]{\textit{#1}.---}
\newcommand{\super}[1]{\ensuremath{^{\textrm{#1}}}}
\newcommand{\sub}[1]{\ensuremath{_{\textrm{#1}}}}
\newcommand{\dC}{\ensuremath{^\circ{\textrm{C}}}}
\newcommand{\tableSubItem}{\addtolength{\leftskip}{1em} \labelitemi \xspace}
\newcommand{\myHangIndent}{\hangindent=5mm}
\newcommand{\myIndent}{\hspace{5mm}}
\newcommand{\ifArg}[2]{\ifthenelse{\equal{#1}{}}{}{#2}\xspace}

\newcommand{\course}[3]{\textbf{#1}\\\textbf{#2:}{#3}}
\newcommand{\gradadvisee}{\ensuremath{^\ddagger}\xspace}
\newcommand{\undergradadvisee}{\ensuremath{^\dagger}\xspace}

%% Create flexible CV item environment
\newlength{\CVItemFirstCol}
\newlength{\CVItemFirstColDefault}
\setlength\CVItemFirstColDefault{2cm}
\newlength{\CVItemSecondCol}
\newcommand{\cvitem}[3]{
    \ifthenelse{\equal{#1}{}}{
        \setlength\CVItemFirstCol{\CVItemFirstColDefault}}{
        \setlength\CVItemFirstCol{#1}}
    \setlength\CVItemSecondCol{\textwidth}
    \addtolength{\CVItemSecondCol}{-\CVItemFirstCol}
    \addtolength{\CVItemSecondCol}{-2\tabcolsep}
    % \begin{tabular}{ @{}p{\CVItemFirstCol} >{\begin{minipage}[t]{\CVItemSecondCol}\raggedright}l<{\end{minipage}}}
    \begin{tabular}{ @{}p{\CVItemFirstCol} >{\begin{minipage}[t]{\CVItemSecondCol}}l<{\end{minipage}}}
         #2 & #3
    \end{tabular}}

%% macro to make long strings breakable over lines
\makeatletter
\def\breakable#1{\xHyphen@te#1$\unskip}
\def\xHyphen@te{\@ifnextchar${\@gobble}{\sw@p{\allowbreak{}\xHyphen@te}}}
% \def\xHyphen@te{\@ifnextchar${\@gobble}{\sw@p{\hskip 0pt plus 1pt\xHyphen@te}}}
\def\sw@p#1#2{#2#1}
\makeatother


\newcommand{\longcontent}[1]{}
% \newcommand{\longcontent}[1]{#1}

\newcommand{\oldcontent}[1]{}
% \newcommand{\oldcontent}[1]{#1}

\newcommand{\docTitle}{Curriculum Vitae\xspace}

%%%%%%%%%%%%%%%%%%%%%%%%%%%%%%%%%%%%%%%%%%%%%%%%%%%%%%%%%%%%%%%%%%%%%%%%%%%%%%%
%%%%%%%%%%%%%%%%%%%%%%%%%%%%%%%%%%%%%%%%%%%%%%%%%%%%%%%%%%%%%%%%%%%%%%%%%%%%%%%
\begin{document}
\singlespacing

\noindent\begin{tabular*}{\textwidth}[tb]{ @{}l @{\extracolsep{\fill}} l@{}}
Assistant Professor \& Curator
& Phone: +1-802-280-5843 \\
Department of Biological Sciences
& E-mail: \href{mailto:joaks@auburn.edu}{joaks@auburn.edu} \\
Museum of Natural History
& Web: \href{http://phyletica.org}{phyletica.org} \\
Auburn University
% & ORCiD: \href{https://orcid.org/0000-0002-3757-3836}{0000-0002-3757-3836} \\
% 101 Rouse Life Sciences Building
& GitHub: \href{https://github.com/joaks1}{joaks1}, \href{https://github.com/phyletica}{phyletica}\\
Auburn, Alabama 36849, USA
& \href{https://scholar.google.com/citations?user=lz3wj6AAAAAJ&hl=en}{Google Scholar} \\
\end{tabular*}

\section*{Professional Appointments}

\bflongcvitem{2022--present}{%
    Senior Manager, Computational Biology, Adaptive Biotechnologies}{%
\begin{cellitemize}
    \item Leader of the Statistics and Algorithms Team in developing
        statistical models and computational tools for inferring disease
        diagnoses and developing therapeutics from patients' genomic and
        transcriptomic data
\end{cellitemize}}

% \bflongcvitem{2016--present}{%
\bflongcvitem{2016--2022}{%
    Assistant Professor \& Curator, Biological Sciences, Auburn University}{%
\begin{cellitemize}
    \item Principal Investigator of a computational evolutionary biology
        research team (11 graduate and 15 undergraduate students, and 3
        postdoctoral researchers) that collects genomic data and develops
        computational tools to test hypotheses about biological
        diversification
    \item Collaboratively use version-control tools for software
        development and project management
\end{cellitemize}}

\bflongcvitem{2013--2016}{%
NSF Postdoctoral Research Fellow, University of Washington}{%
\begin{cellitemize}
    \item Developed statistical methods of comparative phylogeographical
        model choice
\end{cellitemize}}

\section*{Education}
\revcvitem{2013}{
    Ph.D. in Ecology \& Evolutionary Biology, University of Kansas
} \\
\revcvitem{2007}{
    M.S. in Biological Sciences, Louisiana State University
} \\
\revcvitem{2004}{
    B.S. in Biology, University of Wisconsin Oshkosh (\emph{Summa Cum Laude})
}

% \section*{Professional Appointments}
% % \noindent\begin{tabulary}{\textwidth}{ @{} l L @{} }
\cvitem{}{2016--present}{
    Assistant Professor, Department of Biological Sciences, Auburn University
}

% \cvitem{}{2016--present}{
%     Curator of Amphibians and Reptiles, Auburn University Museum of Natural
%     History
% }

\cvitem{}{2013--2016}{
    NSF Postdoctoral Research Fellow, University of Washington
    \longcontent{
    \begin{itemize}
        \item Developed statistical methods of comparative phylogeographical
            model choice
    \end{itemize}
    }
}

\oldcontent{
\cvitem{}{2010--2013}{
    Research Assistant, University of Kansas
    \begin{itemize}
        \item Developed phylogenetic software with Dr.\ Mark Holder
    \end{itemize}
}

% \cvitem{}{2007--2009}{
%     Curatorial Assistant, University of Kansas Biodiversity Institute,
%     Herpetology Collection
%     \begin{itemize}
%         \item Performed full duties of Herpetology Collections Manager
%     \end{itemize}
% }

\cvitem{}{2008--2009}{
    Research Assistant, University of Kansas, NSF EPSCoR (EPS-0553722)
    \begin{itemize}
        \item Collected herpetological specimens, ecological data, and genetic
            samples across the Great Plains
    \end{itemize}
}

% \cvitem{}{2006--2007}{
%     Curatorial Assistant, Louisiana State University Museum of Natural Science,
%     Herpetology Collection
%     \begin{itemize}
%         \item Performed full duties of Herpetology Collections Manager
%     \end{itemize}
% }

% \cvitem{}{2004--2006}{
%     Research Assistant, Louisiana State University Museum of Natural Science
%     \begin{itemize}
%         \item Geo-referenced specimen localities for NSF-funded HerpNET
%             project
%     \end{itemize}
% }

\cvitem{}{2002--2004}{
    Undergraduate Research Assistant, University of Wisconsin Oshkosh
    \begin{itemize}
        \item Analyzed long-term ecological data collected by Dr.\ Gregory
            Adler
    \end{itemize}
}

\cvitem{}{2002}{
    Undergraduate Research Assistant, University of Wisconsin Oshkosh
    \begin{itemize}
        \item Conducted immunological study on alligators with Dr.\ Colleen
            McDermott
    \end{itemize}
}

\cvitem{}{2001}{
    Undergraduate Researcher, University of Wisconsin Oshkosh
    \begin{itemize}
        \item Conducted population study of turtles and their parasites in
            Wisconsin
    \end{itemize}
}
}

% \end{tabulary}



% \section*{Curatorial Experience}
% \cvitem{}{2016--present}{
    Curator of Amphibians and Reptiles, Auburn University Museum of Natural
    History
}

\cvitem{}{2007--2009}{
    Curatorial Assistant, University of Kansas Biodiversity Institute,
    Herpetology Collection
    % \begin{itemize}
    %     \item Performed full duties of Herpetology Collections Manager
    % \end{itemize}
}

\cvitem{}{2006--2007}{
    Curatorial Assistant, Louisiana State University Museum of Natural Science,
    Herpetology Collection
    % \begin{itemize}
    %     \item Performed full duties of Herpetology Collections Manager
    % \end{itemize}
}

\cvitem{}{2004--2006}{
    Research Assistant, Louisiana State University Museum of Natural Science
    \begin{itemize}
        \item Geo-referenced specimen localities for NSF-funded HerpNET
            project
    \end{itemize}
}


\section*{Select Grants Awarded (\$1.16 million total)}
% \myHangIndent
Grismer, J.L., J.R.\ Oaks, R.M.\ Brown, N.B.\ Ananjeva, X.\ Guo, and N.\
Batsaikhan.
Revisiting Prezewalski's Expeditions: Population genetics in the Gobi Desert.
National Geographic Committee for Research and Exploration Grant.
\$20,000.
5/2014--6/2015.
Funded.

\myHangIndent
Oaks, J.R.\ (Co-PI), R.\ Brown (PI), and M.\ Holder (Co-PI).
Comparative Phylogeography of a Dynamic Archipelago.
National Science Foundation Doctoral Dissertation Improvement Grant.
\$14,886.
8/2010--8/2012.
Funded.

\myHangIndent
Oaks, J.R.
Comparative Phylogeography of a Dynamic Archipelago.
KU Biodiversity Institute Panorama Grant.
\$1000.
5/2011--5/2012.
Funded.

\myHangIndent
Oaks, J.R.
Comparative Phylogeography of a Dynamic Archipelago.
KU Office of Graduate Studies Doctoral Student Research Award.
\$2000.
7/2011--7/2012.
Funded.

\ignore{
\myHangIndent
Oaks, J.R.
Comparative Phylogeography of a Dynamic Archipelago.
The Society for the Study of Evolution Rosemary Grant Graduate Research Award.
\$2,085.
7/2010--7/2011.
Not funded.
}
\myHangIndent
Oaks, J.R.
NSF stipend to attend a five-day short course, ``Comparative Methods and
Macroevolution In R.''
Stipend covered travel and lodging; approximately \$1,000.
6/2010.
Funded.

\myHangIndent
Oaks, J.R.
KU Department of Ecology and Evolutionary Biology Travel Grant.
\$300.
6/2010.
Funded.

\myHangIndent
Oaks, J.R.
KU Biodiversity Institute Travel Grant.
\$300.
6/2010.
Funded.

\myHangIndent
Oaks, J.R.
Comparative Phylogeography of Southern Indochina.
Society of Systematic Biologists Graduate Student Award.
\$1650.
6/2009--6/2010.
Funded.

\myHangIndent
Oaks, J.R.
Comparative Phylogeography of Southern Indochina.
Sigma Xi.
\$1000.
6/2009--6/2010.
Funded.

\ignore{
\myHangIndent
Oaks, J.R.
Comparative Phylogeography of Southern Indochina.
The Society for the Study of Amphibians and Reptiles Dean E. Metter Award.
\$800.
6/2009--6/2010.
Not funded.
}
\myHangIndent
Oaks, J.R.
KU Department of Ecology and Evolutionary Biology Travel Grant.
\$300.
5/2010.
Funded.

\myHangIndent
Oaks, J.R.
Estimating Species Trees Workshop Travel Grant.
University of Michigan Museum of Zoology.
\$500.
1/2009.
Funded.

\myHangIndent
Oaks, J.R.
A Six-day Course in Statistical Phylogeography.
KU Biodiversity Institute Panorama Grant.
\$900.
4/2009.
Funded.

\myHangIndent
Oaks, J.R.
A Herpetofaunal Survey and Biogeographic Study of the Cardamom Mountains in
Southeast Asia.
Louisiana State University BioGrads Grant.
\$300.
5/2007--7/2007.
Funded.

\myHangIndent
Oaks, J.R.
A Herpetofaunal Survey of Phnom Samkos, Cambodia.
Louisiana State University Museum of Natural Science Graduate Student Research
Grant.
\$600.
5/2007--7/2007.
Funded.

\myHangIndent
Oaks, J.R.
The Biogeography of the Herpetofauna of the Cardamom Mountains.
Louisiana State University BioGrads Grant.
\$300.
5/2006--7/2006.
Funded.

\myHangIndent
Oaks, J.R.
A Herpetofaunal Survey of the Cardamom Mountains, Cambodia.
Louisiana State University Museum of Natural Science Graduate Student Research
Grant.
\$300.
5/2006--7/2006.
Funded.

\myHangIndent
Oaks, J.R.
Herpetology Fieldwork in the Cardamom Mountains of Southeast Asia.
Louisiana State University BioGrads Grant.
\$300.
5/2006--7/2006.
Funded.

\myHangIndent
Oaks, J.R.
Geographic Variation of Multiple Paternity in the American alligator
(\emph{Alligator mississippiensis}): Does Selection Rescue Genetic Diversity?
Sigma Xi.
\$1000.
1/2005--12/2005.
Funded.

\myHangIndent
Oaks, J.R.
Coevolution Among Turtles and Parasites: A Test of the Tenant.
University of Wisconsin Oshkosh Undergraduate Research Grant.
\$2500.
6/2001--6/2002.
Funded.


\myHangIndent
\textbf{Oaks, J.R.\ (PI)}.
Senior personnel: L.L.\ Grismer, C.D.\ Siler, and P.L.\ Wood, Jr.
Generalizing Bayesian Phylogenetics to Infer Shared Evolutionary Events.
National Science Foundation Division of Environmental Biology Award
\href{https://www.nsf.gov/awardsearch/showAward?AWD_ID=1656004&HistoricalAwards=false}{1656004}.
5/2017--5/2022.
\$551,169.
Funded.

\myHangIndent
Zohdy, S.\ (Co-PI), T.\ Schwartz (Co-PI), \textbf{J.R.\ Oaks (Co-I)}, and Z.\ Farris (Co-I).
The Coevolution Effect: A Mechanism for Viral Spillover into Humans.
Auburn University Intramural Grants Program.
5/2017--5/2019.
\$40,000.
Funded.

% \section*{Select Grants Pending}
% % \myHangIndent
\textbf{Oaks, J.R.\ (PI)}.
A process-based, modular, phylogenetic framework for inferring rates
and patterns of shared evolutionary events.
National Science Foundation Division of Environmental Biology
9/2024--9/2028.
\$961,416.
Pending.

% \myHangIndent
% \textbf{Oaks, J.R.\ (PI)}.
% CAREER: A process-based, modular, phylogenetic framework for inferring rates
% and patterns of shared evolutionary events.
% National Science Foundation Division of Environmental Biology
% 6/2022--6/2027.
% \$1,287,213.
% Pending.

\section*{Fellowships}
\cvitem{3em}{2013}{
    National Science Foundation Postdoctoral Research Fellowship in Biology
}

\cvitem{3em}{2013}{
    National Research Council Postdoctoral Research Associateship---National
    Institute of Standards and Technology (Awarded, but declined)
}

\longcontent{
\myHangIndent
Improving methods for testing models of shared evolutionary history.
KU Office of Graduate Studies Summer Research Fellowship.
\$5,000.
5/2013--8/2013.

\myHangIndent
Kenneth B. Armitage Award for Excellence in Teaching.
BIOL 350---Introduction to Genetics.
\$250.
5/2011.

\ignore{
\myHangIndent
Comparative Phylogeography of a Dynamic Archipelago.
KU Office of Graduate Studies Summer Research Fellowship.
\$5,000.
5/2010--8/2010.
Not funded.
}

\myHangIndent
Comparative Phylogeography of a Dynamic Archipelago.
KU Department of Ecology and Evolutionary Biology Summer Fellowship.
\$4,000.
5/2010--8/2010.

\myHangIndent
Henri Seibert award for best paper in Systematics/Evolution.
\$300.
Joint Meeting of Ichthyologists and Herpetologists, Portland, Oregon, 07/09.

\myHangIndent
Graduate Student Enhancement.
\$15,000 (3 years).
Louisiana State University Department of Biological Sciences.
2004--2007.

\myHangIndent
Outstanding Biology Major Award.
\$500.
University of Wisconsin Oshkosh.
2004.

\myHangIndent
Leslie-Allen Fellowship in Ecology and Field Biology.
\$500.
University of Wisconsin Oshkosh.
2003.

\myHangIndent
Leslie-Allen Fellowship in Ecology and Field Biology.
\$500.
University of Wisconsin Oshkosh.
2002.
}



% Publications
\section*{Select Publications (\href{https://scholar.google.com/citations?user=lz3wj6AAAAAJ&hl=en}{41 total; 1887 citations})}
\ugsymbol{}mentored undergraduate student;
\phdsymbol{}mentored graduate student;
\postdocsymbol{}mentored postdoc.

\nocite{*}
\printbibliography[filter=resumepapers, heading=none]

\section*{Select Presentations (18 invited seminars and 41 conference presentations)}
Presenter in italics;
\ugsymbol{}mentored undergraduate student;
\phdsymbol{}mentored graduate student;
\postdocsymbol{}mentored postdoc.

\nocite{*}
\printbibliography[filter=highlightedpresentations, heading=none]

% \longcontent{
\section*{Teaching Experience}
% \cvitem{6em}{Fall 2020}{
    Advanced statistical methods for evolutionary genetics (BIOL 7960),
    Auburn University
}

\cvitem{6em}{Fall 2020}{
    Evolution \& Systematics (BIOL 3030),
    Auburn University
}

\cvitem{6em}{Summer 2020}{
    Advanced statistical methods for evolutionary genetics (BIOL 7960),
    Auburn University
}

\cvitem{6em}{Spring 2020}{
    Herpetology (BIOL 5740/6740),
    Auburn University
}

\cvitem{6em}{Fall 2019}{
    Evolution \& Systematics (BIOL 3030),
    Auburn University
}

\cvitem{6em}{Summer 2019}{
    Workshop on basic scripting using version control to improve
    reproducibility in science,
    Auburn Bioinformatics Bootcamp,
    Auburn University
}

\cvitem{6em}{Spring 2019}{
    Scripting for Biologists (BIOL 7180),
    Auburn University
}

\cvitem{6em}{Fall 2018}{
    Evolution \& Systematics (BIOL 3030),
    Auburn University
}

\cvitem{6em}{Summer 2018}{
    Advanced statistical methods for evolutionary genetics (BIOL 7960),
    Auburn University
}

\cvitem{6em}{Summer 2018}{
    Workshop on using version control to improve reproducibility in science,
    Auburn Bioinformatics Bootcamp,
    Auburn University
}

\cvitem{6em}{Spring 2018}{
    Herpetology (BIOL 5740/6740),
    Auburn University
}

\cvitem{6em}{Summer 2017}{
    Advanced statistical methods for evolutionary genetics (BIOL 7960),
    Auburn University
}

\cvitem{6em}{Summer 2017}{
    Workshop on using version control to improve reproducibility in science,
    Auburn Bioinformatics Bootcamp,
    Auburn University
}

\cvitem{6em}{Spring 2017}{
    Herpetology (BIOL 5740/6740),
    Auburn University
}

\cvitem{6em}{Fall 2016}{
    Evolution \& Systematics (BIOL 3030),
    Auburn University
}

\cvitem{6em}{Fall 2016}{
    Evolution (BIOL 7200),
    Auburn University
}

\cvitem{6em}{Spring 2015}{
    Lecturer, Introductory Biology (BIOL 180),
    University of Washington
}

\cvitem{6em}{Fall 2014}{
    Lecturer, Introductory Biology (BIOL 180),
    University of Washington
    % \longcontent{
    % \begin{itemize}
    %     \item Developing and implementing active-learning exercises, assessing
    %         student learning, and lecturing under the mentorship of Dr.\ Scott
    %         Freeman.
    % \end{itemize}
    % }
}

\longcontent{
\cvitem{6em}{2014}{
    Guest lecturer, Biogeography (BIOL 470; primary instructor John Klicka),
    University of Washington
    % \begin{itemize}
    %     \item Presented lecture on Phylogenetics
    % \end{itemize}
}

\cvitem{6em}{2014}{
    Applied Phylogenetics (BIOL 449; primary instructor Adam Leach\'{e}) for
    undergraduate and graduate students, University of Washington
    \begin{itemize}
        \item Developed learning-assessment survey for the course
        \item Led guest lecture and lab on divergence-time estimation
    \end{itemize}
}

\cvitem{6em}{2013}{
    Teaching Assistant for Software Carpentry Bootcamp at the University of
    Kansas, 22--23 August 2013
    % \begin{itemize}
    %     \item Assisted in teaching graduate students, postdocs, and faculty
    %         basic principles of computer programming
    % \end{itemize}
}

\cvitem{6em}{2013}{
    Instructor for Workshop on New Methods for Phylogenomics and Metagenomics,
    University of Texas at Austin, 17 February 2013
    % \begin{itemize}
    %     \item Taught workshop for the software package
    %         \href{http://phylo.bio.ku.edu/software/sate/sate.html}{{SAT}\'{e}}
    %         to approximately 150 participants, including graduate students,
    %         postdocs, and faculty
    % \end{itemize}
}

\cvitem{6em}{2012}{
    Teaching Assistant for Workshop on Advances in Multiple Sequence Alignment
    and Phylogeny Estimation, Smithsonian Institution, Washington DC, 20--21
    May 2012
    % \begin{itemize}
    %     \item Helped write tutorials and assisted workshop participants with
    %         phylogenetic analyses
    % \end{itemize}
}

\oldcontent{
\cvitem{6em}{2010}{
    Teaching Assistant, Introduction to Genetics (BIOL 350), University of
    Kansas (two semesters)
    \begin{itemize}
        \item Led weekly discussion sections (lectures and problem solving)
        \item Led exam review sessions
        \item Received Award for Excellence in Teaching
    \end{itemize}
}

\cvitem{6em}{2010}{
    Guest Lecturer, Graduate-level Evolutionary Biology (BIOL 712; primary
    instructor John Kelly), University of Kansas
    \begin{itemize}
        \item Presented lecture on Statistical Phylogeography
    \end{itemize}
}

\cvitem{6em}{2009}{
    Teaching Assistant, Introduction to Biostatistics (BIOL 570), University of
   Kansas
    \begin{itemize}
        \item Led weekly computer lab sections
    \end{itemize}
}

\cvitem{6em}{2006}{
    Teaching Assistant, Herpetology (BIOL 4146), Louisiana State University
    \begin{itemize}
        \item Developed, organized, and presented weekly labs
    \end{itemize}
}

\cvitem{6em}{2003--2004}{
    Tutor for the University of Wisconsin Oshkosh Student Support Services
}
}
}


% \begin{cvitemize}
%     \item Since 2014, I have taught a number of undergraduate and graduate
%         courses, including
%         Scripting for Biologists,
%         Advanced Statistical Methods for Evolutionary Genetics,
%         Evolution,
%         and
%         Introductory Biology
%         % Herpetology,
%     \item Taught workshop modules on basic scripting and using version control
%         to improve reproducibility in science
% \end{cvitemize}
\longcvitem{2016--present}{%
    Auburn University}{%
\begin{cellitemize}
    \item Taught a number of undergraduate and graduate courses, including
        \href{http://phyletica.org/teaching/s4b}{Scripting for Biologists},
        Advanced Statistical Methods for Evolutionary
        Genetics, and Evolution
\end{cellitemize}}
\longcvitem{2012--present}{%
    Workshop Instructor}{%
\begin{cellitemize}
    \item Instructor for workshops (e.g., Software Carpentry and Auburn
        Bioinformatics Bootcamp); topics include basic scripting and using
        version control to improve reproducibility in science
\end{cellitemize}}
\longcvitem{2014--2016}{%
    Lecturer, Introductory Biology (BIOL 180),
    University of Washington}{%
\begin{cellitemize}
        \item Developed and implemented active-learning exercises, assessed
            student learning, and lectured in collaboration with the
            Biology Education Research Group (BERG)
\end{cellitemize}}
% }

% % \longcontent{
% \section*{Advisee Honors and Awards}
% \cvitem{3em}{2019}{
    \textbf{Tanner Myers}: Auburn University Presidential Graduate Research Fellowship (\$90,000).
}
\cvitem{3em}{2019}{
    \textbf{Claire Tracy}: Auburn University Presidential Graduate Research Fellowship (\$90,000).
}
\cvitem{3em}{2019}{
    \textbf{Claire Tracy}: Auburn University Allsup Fellowship (\$5,000).
}
\cvitem{3em}{2019}{
    \textbf{Randy Klabacka}: Henry Seibert Award
    for Best Student Oral Presentation in Evolution/Systematics
    at the Joint Meeting of Ichthyologists and Herpetologists.
}
\cvitem{3em}{2019}{
    \textbf{Kyle David (advisor: Ken Halanych)}: David and Marvalee Wake Award
    for Best Student Oral Presentation at Annual Meeting of the Society for
    Integrative \& Comparative Biology
    (for presenting work that began during his rotation through my lab in
    Spring 2018)
}
\cvitem{3em}{2018}{
    \textbf{Breanna Sipley}: National Science Foundation Graduate Research
    Fellowship (\$138,000)
}
\cvitem{3em}{2017}{
    \textbf{Aundrea Westfall}: Cellular and Molecular Biosciences Peaks of
    Excellence Research Fellowship (\$6,900)
}
\cvitem{3em}{2017}{
    \textbf{Breanna Sipley}: Stanley W.\ Watson Foundation Education Fund to
    attend 2017 Workshop on Molecular Evolution, Marine Biological Laboratory,
    Woods Hole, Massachusetts (\$1,502.50)
}
\cvitem{3em}{2017}{
    \textbf{Breanna Sipley}: American Society of Naturalists Travel Award to
    attend 2017 Joint Evolution meeting, Portland, Oregon (\$500)
}
\cvitem{3em}{2017}{
    \textbf{Breanna Sipley}: Society of Systematic Biologists Travel Award to
    attend TreeScaper workshop and SSB Standalone Meeting, Baton Rouge,
    Louisiana (\$500)
}
\cvitem{3em}{2017}{
    \textbf{Breanna Sipley}: Auburn University Museum of Natural History
    Meredith Ann Birchfield Endowed Fund for Excellence (\$600)
}
\cvitem{3em}{2017}{
    \textbf{Randy Klabacka}: Auburn University Museum of Natural History
    Meredith Ann Birchfield Endowed Fund for Excellence (\$1500)
}
\cvitem{3em}{2017}{
    \textbf{Randy Klabacka}: Cellular and Molecular Biosciences Peaks of
    Excellence Research Fellowship (\$6,900)
}

% % }

% \longcontent{
\section*{Select Software Packages}
% \myHangIndent
\href{https://github.com/phyletica/ecoevolity}{Ecoevolity}:
    Estimating evolutionary coevality.
    Author: J.\ Oaks.

\myHangIndent
\href{https://github.com/phyletica/sumcoevolity}{Pycoevolity}:
    A python package for summarizing the output of Ecoevolity.
    Author: J.\ Oaks.

\myHangIndent
\href{https://github.com/joaks1/fragcoalsim}{Fragcoalsim}:
    A package for coalescent simulations and expectations under population
    fragmentation.
    Author: J.\ Oaks.

\myHangIndent
\href{https://github.com/phyletica/SumTimes}{SumTimes}:
    A tool for estimating the probability of divergence-time scenarios from
    posterior samples of trees.
    Author: J.\ Oaks.

\myHangIndent
\href{https://github.com/joaks1/PyMsBayes}{PyMsBayes}:
    Multi-processing wrapper and API for approximate-Bayesian
    computation.
    Author: J.\ Oaks.

\myHangIndent
\href{https://github.com/joaks1/dpp-msbayes}{dpp-msbayes}:
    Approximate-Bayesian method for comparative phylogeographical model choice.
    Author: J.\ Oaks.

\myHangIndent
\href{https://github.com/joaks1/abacus}{ABACUS}:
    Approximate BAyesian C UtilitieS.
    Author: J.\ Oaks.

\myHangIndent
\href{https://github.com/joaks1/SeqSift}{SeqSift}:
    Package for manipulating molecular sequence data.
    Author: J.\ Oaks.

\myHangIndent
\href{https://github.com/joaks1/SplitFreq}{SplitFreq}:
    A tool to compare split frequencies between collections of trees.
    Author: J.\ Oaks.

\myHangIndent
\href{http://phylo.bio.ku.edu/software/sate/sate.html}{{SAT}\'{e}}:
    Phylogenetic software for simultaneous alignment and tree estimation.
    Authors: J.\ Yu, M.T.\ Holder, J.\ Sukumaran, S.\ Mirarab, and J.\ Oaks.

\noindent
\href{https://github.com/phyletica/ecoevolity}{Ecoevolity}:
    Estimating evolutionary coevality. \\
\noindent
\href{https://github.com/phyletica/sumcoevolity}{Pycoevolity}:
    A python package for summarizing the output of Ecoevolity. \\
\noindent
\href{https://github.com/joaks1/fragcoalsim}{Fragcoalsim}:
    A package for coalescent simulations and expectations under population
    fragmentation. \\
\noindent
\href{http://phylo.bio.ku.edu/software/sate/sate.html}{{SAT}\'{e}}:
    Phylogenetic software for simultaneous alignment and tree estimation.
% }

% % \longcontent{
% \section*{Websites}
% \begin{tightItemize}
%     \item Phyletica Lab website: \url{http://phyletica.org}
%     \item Phyletica Lab YouTube Channel: \url{https://www.youtube.com/channel/UCPPWy11zYqq-9lE_CQ0qPrw}
%     \item Ecoevolity Documentation: \url{http://phyletica.org/ecoevolity}
%     \item PyMsBayes Documentation: \url{http://joaks1.github.io/PyMsBayes}
%     % \item SumTimes Documentation: \url{http://phyletica.github.io/SumTimes}
% \end{tightItemize}
% % }

% \section*{Open-Science Notebooks}
% \ugsymbol{}mentored undergraduate student;
% \phdsymbol{}mentored graduate student;
% \postdocsymbol{}mentored postdoc.
% \nocite{*}
% \printbibliography[filter=openscinotebooks, heading=none]

% \longcontent{
% \section*{Professional Affiliations}
% \begin{myItemize}
\item American Society of Ichthyologists and Herpetologists
\item International Biogeography Society
\item Society for the Study of Amphibians and Reptiles
\item Society for the Study of Evolution
\item Society of Systematic Biologists
\end{myItemize}


% }

% \oldcontent{
% \section*{Workshops Attended}
% \myHangIndent
Comparative Methods and Macroevolution in R (five-day course).
National Center for Ecological Analysis Synthesis, Santa Barbara, California.
17--21 June 2010.

\myHangIndent
Statistical Phylogeography Course.
American Museum of Natural History Southwestern Research Station, Arizona.
5--10 April 2009.

\myHangIndent
Estimating Species Trees: A Phylogenetic Paradigm for the 21st Century.
Unversity of Michigan, Ann Arbor, Michigan.
10--11 January 2009.

\myHangIndent
HerpNET Georeferencing Workshop.
University of Kansas, Lawrence, Kansas.
16--18 December 2004.


% }

% \section*{Service}
% \subsection*{University Service}
% \cvitem{6em}{2016--present}{DBS: AUMNH Curator of Reptiles \& Amphibians}

\cvitem{6em}{2018--2021}{AU: Member of University Senate Computing Committee}

\cvitem{6em}{2019--present}{DBS: Committee member of the DBS Awards Committee}

\cvitem{6em}{2020}{DBS: Ad hoc contributor to the assessment tools for Student
    Learning Outcome (SLO) for Evolution (DBS SLO 7); attended committee
    meetings and provided assessment questions, rubrics, and feedback}

\cvitem{6em}{2020}{COSAM: Ad hoc award selection committee}

% L.S.
\cvitem{6em}{2020}{DBS: Peer reviewer of teaching of DBS faculty member}

% D.W.
\cvitem{6em}{2019}{DBS: Peer reviewer of teaching of DBS faculty member}

\cvitem{6em}{2018--2019}{DBS: Member of search committee for a Museum Director}

\cvitem{6em}{2018--2019}{COSAM: Member of search committee for a faculty position in
    Biostatistics/Bionformatics}

\cvitem{6em}{2018--2019}{DBS: Member of committee assigned with implementing
    new Student Learning Outcome (SLO) for written communication (DBS SLO 4)}

\cvitem{6em}{2017--2019}{DBS: Committee member of the DBS Graduate Studies Committee}

\cvitem{6em}{2016--2018}{DBS: Committee member of the DBS Informatics Steering Group}

\cvitem{6em}{2018}{COSAM: Member of search committee for Specialist I position in
    Communications and Marketing}

\cvitem{6em}{2018}{AU: Snake safety training at E.W.\ Shell Fisheries
    Center; 61 participants, including undergraduate and graduate
    students, postdocs, faculty, and staff}

\cvitem{6em}{2017}{AU: Snake safety training at E.W.\ Shell Fisheries Center}


% \subsection*{Professional Service}
% \cvitem{6em}{2021--present}{
Associate editor for \href{https://academic.oup.com/sysbio}{\emph{Systematic Biology}}
}

\cvitem{6em}{2021}{
Conduct moderator and Evo Ally for the 2021 Evolution Meeting
}

\cvitem{6em}{2021}{
Participated in the Student-Faculty Networking Lunch Program at the 2021
Evolution Meeting
}

\cvitem{6em}{2021}{
Reviewer for the
    \emph{Society of Systematic Biologists Graduate Student Research Awards}
}

\cvitem{6em}{2020}{
    Attended Safe-Zone Training. Offered by Auburn University's Human Resources
    Development
}

\cvitem{6em}{2020}{
    Attended Workshop on Gender Identity and Pronouns for Faculty. Offered by
    Auburn University's Office of Inclusion and Diversity and the College of
    Education's Critical Studies Working Group.
}


\cvitem{6em}{2020}{
Organized and participated in a Student-Faculty Lunch as part of the Evolution
Community Resources for Early Career Researchers (ECR\super{2}).
}

\cvitem{6em}{2020}{
Reviewer for the
    \emph{Society of Systematic Biologists Graduate Student Research Awards}
}

\cvitem{6em}{2019}{Hosted a booth at the iEvoBio Software Bazaar. iEvoBio
    Conference. Providence, Rhode Island.}

\cvitem{6em}{2016--2021}{
Member of the editorial board for \emph{Systematic Biology}
}

\cvitem{6em}{2020}{
Reviewer for the
    \emph{Society of Systematic Biologists Graduate Student Research Awards}
}

\cvitem{6em}{2019}{
Reviewer for the
    \emph{Society of Systematic Biologists Mini-ARTS Program}
}

\cvitem{6em}{2018}{Lead Instructor of a Comparative Phylogeography Workshop. Standalone Meeting of the Society of Systematic Biologists. Columbus, Ohio. 52 participants, including graduate students, postdocs, and professors. \url{http://phyletica.org/ecoevolity/workshops/ssb-2018.html}}

\cvitem{6em}{2018}{
Reviewer for the
    \emph{Society of Systematic Biologists Graduate Student Research Awards}
}

\cvitem{6em}{2017}{
Reviewer for the
    \emph{Society of Systematic Biologists Graduate Student Research Awards}
}

\cvitem{6em}{2016}{
Reviewer for the
    \emph{National Science Foundation Dimensions of Biodiversity Program}
}

\cvitem{6em}{2016}{
Reviewer for the
    \emph{Society of Systematic Biologists Mini-ARTS Program}
}

\cvitem{6em}{2014}{
Reviewer for the
    \emph{National Security Agency Mathematical Sciences Grant Program}
}

\myHangIndent
% Ad hoc reviewer for
Reviewer for
    \emph{BMC Evolutionary Biology} (3),
    \emph{Evolution} (5),
    \emph{Molecular Biology and Evolution} (1),
    \emph{Molecular Ecology} (3),
    \emph{Molecular Phylogenetics and Evolution} (1),
    \emph{Organisms Diversity and Evolution} (2),
    \emph{Proceedings of the Royal Society B} (2),
    \emph{Systematic Biology} (17),
    and
    \emph{Theoretical Population Biology} (1)

\longcontent{
\cvitem{6em}{2011}{
Graduate student representative on the departmental committee for
    genomics faculty hire, University of Kansas Department of Ecology and
    Evolutionary Biology
}

\cvitem{6em}{2006--2007}{
Seminar Coordinator, Louisiana State University Museum of Natural Science
    Seminar Series
}
}


% \section*{Outreach}
% \myHangIndent
Blog contributor:
\href{http://phyletica.org/posts/}{phyletica.org} (2013--present)

\myHangIndent
Contributed a short story about my research to
\href{http://sciworthy.com/science-news/science-authors/whos-the-living-fossil/}{Sciworthy.com},
a news site with the goal of improving public understanding of science (2014)

\myHangIndent
While conducting fieldwork in Malaysia, I helped train undergraduate and
graduate students from Universiti Kebangsaan Malaysia, Universiti Sains
Malaysia, and La Sierra University in methods of field collection and specimen
preparation (Fall 2011)

\myHangIndent
Trained two local high school student volunteers as fieldwork and curatorial
assistants for the University of Kansas NSF EPSCoR project (2008)

\myHangIndent
Louisiana State University Museum of Natural Science Children's Special
Saturday (2006)
\begin{itemize}
    \item Gave a presentation to children, ages 5--10, about the basic biology
        and diversity of frogs
\end{itemize}

\myHangIndent
Louisiana State University Museum of Natural Science Children's Special
Saturday (2005)
\begin{itemize}
    \item Introduced high school students to the imperative role of scientific
        research collections and curatorial methods
\end{itemize}



% \longcontent{
\section*{Additional Research Experience \& Skills}
% \myHangIndent
{\sffamily\itshape Computer programming languages:} Python, R, C/C++, Java,
PERL, shell scripting, and \LaTeX

\myHangIndent
{\sffamily\itshape Fieldwork:} Myanmar (2018); Cambodia (2012, 2006); Malaysia (2011, 2006);
Great Plains, USA (2007--2009); Southeastern USA (2016--present \& 2004--2007); Florida, USA
(2002); Wisconsin, USA (2001)

\myHangIndent
{\sffamily\itshape Computer programming languages:} C/C++, Python, Julia, R, Java,
PERL, shell scripting, and \LaTeX
% }

\section*{Select Service \& Outreach Activities}

\bfrevcvitem{2021--present}{
Associate editor for \href{https://academic.oup.com/sysbio}{\emph{Systematic Biology}}
} \\
\bfrevcvitem{2021}{
Conduct moderator and Evo Ally for the 2021 Evolution Meeting
} \\
\bflongcvitem{2017--present}{Prison education---Alabama Prison Arts + Education Project (APAEP)}{
    \begin{cellitemize}
        \item Taught 14-week evolutionary biology courses to diverse classes of
            students incarcerated in Alabama Correctional Facilities
        \item Created a biology-themed issue of \emph{The Warbler} newsletter,
            which is distributed by the APAEP to over 800 current and former
            students, and by other prison-education organizations across 28
            states and 3 countries
    \end{cellitemize}
}
\bflongcvitem{2020--2021}{Diversity, Equity, and Inclusion Training}{
    \begin{cellitemize}
        \item Completed Safe-Zone Training, offered by Auburn University's
            Human Resources Development
        \item Attended Workshop on Gender Identity and Pronouns, offered by
            Auburn University's Office of Inclusion and Diversity
    \end{cellitemize}
}
\bflongcvitem{2019}{\href{https://www.auburn.edu/cosam/departments/diversity/summerbridge/index.htm}{STEM Summer Bridge Program}}{
    \begin{cellitemize}
        \item Lecturer for the program to help prepare students from
            historically excluded groups for academia
    \end{cellitemize}
}
\bflongcvitem{2018 \& 2019}{Alabama State Science Olympiad}{
    \begin{cellitemize}
        \item Developed and administered practical exam in biology for the
            middle and high school competitions
    \end{cellitemize}
}
\bflongcvitem{2016--2018}{Committee member of the Informatics Steering Group (Auburn University)}{
    \begin{cellitemize}
        \item Developed a curriculum for
            graduate students to earn a
            \href{http://bulletin.auburn.edu/thegraduateschool/graduatedegreesoffered/biologicalsciencesmsphd_major/computationalbiology_gradcert/}{Graduate
                Certificate in Computational Biology}.
    \end{cellitemize}
}
\bflongcvitem{2016}{Project Lead the Way (PLTW) Alabama State Conference}{
    \begin{cellitemize}
        \item Gave a presentation about bioinformatics to K-12 educators from
            across Alabama
    \end{cellitemize}
}
% \myHangIndent
% Ad hoc reviewer for
\bflongcvitem{}{Peer review}{
    \begin{cellitemize}
        \item Reviewer of grant proposals for the
National Science Foundation Dimensions of Biodiversity Program
and
National Security Agency Mathematical Sciences Grant Program
% Society of Systematic Biologists Graduate Student Research Awards
        \item Reviewer of manuscripts for
\emph{BMC Evolutionary Biology}, % (3),
\emph{Evolution}, % (5),
\emph{Molecular Biology and Evolution}, % (1),
\emph{Molecular Ecology}, % (3),
\emph{Molecular Phylogenetics and Evolution}, % (1),
\emph{Organisms Diversity and Evolution}, % (2),
\emph{Proceedings of the Royal Society B}, % (2),
\emph{Systematic Biology}, % (13),
and
\emph{Theoretical Population Biology}% (1)
    \end{cellitemize}
}


% \section*{References}
% {\raggedright
% Rafe Brown \\
\myIndent Curator of Herpetology, Biodiversity Institute \\
\myIndent Associate Professor, Department of Ecology \& Evolutionary Biology \\
\myIndent University of Kansas \\
\myIndent Dyche Hall, 1345 Jayhawk Boulevard \\
\myIndent Lawrence, KS 66045 \\
\myIndent (785) 864-3403 \\
\myIndent \href{mailto:rafe@ku.edu}{\tt rafe@ku.edu}

Scott Freeman \\
\myIndent Principal Lecturer, Department of Biology \\
\myIndent University of Washington \\
\myIndent Box 351800 \\
\myIndent Seattle, WA 98195\\
\myIndent (206) 300-4448 \\
\myIndent \href{mailto:srf991@u.washington.edu}{\tt srf991@u.washington.edu}

Mark Holder \\
\myIndent Associate Professor, Department of Ecology \& Evolutionary Biology \\
\myIndent University of Kansas \\
\myIndent 1200 Sunnyside Avenue \\
\myIndent Lawrence, KS 66045 \\
\myIndent (785) 856-2873 \\
\myIndent \href{mailto:mtholder@ku.edu}{\tt mtholder@ku.edu}

% Lee Grismer \\
% \myIndent Professor, Department of Biology \\
% \myIndent La Sierra University \\
% \myIndent 4500 Riverwalk Parkway \\
% \myIndent Riverside, CA 92515 \\
% \myIndent (951) 785-2345 \\
% \myIndent \href{mailto:lgrismer@lasierra.edu}{\tt lgrismer@lasierra.edu}

% Adam Leach\'{e} \\
% \myIndent Assistant Professor, Department of Biology \\
% \myIndent University of Washington \\
% \myIndent Box 351800 \\
% \myIndent Seattle, WA 98195\\
% \myIndent (206) 543-7622 \\
% \myIndent \href{mailto:leache@uw.edu}{\tt leache@uw.edu}

% Vladimir Minin \\
% \myIndent Associate Professor, Departments of Statistics and Biology\\
% \myIndent University of Washington \\
% \myIndent Box 354322 \\
% \myIndent Seattle, WA 98195\\
% \myIndent (206) 543-4302 \\
% \myIndent \href{mailto:vminin@uw.edu}{\tt vminin@uw.edu}

% Cameron Siler \\
% \myIndent Assistant Curator of Herpetology, Sam Noble Museum \\
% \myIndent Assistant Professor, Department of Biology \\
% \myIndent University of Oklahoma \\
% \myIndent 2401 Chautauqua Avenue \\
% \myIndent Norman, OK 73072\\
% \myIndent (405) 325-3718 \\
% \myIndent \href{mailto:camsiler@ou.edu}{\tt camsiler@ou.edu}


% }

% \section*{International Collaborators}

\subsection*{Computational methods development}

\cvitem{10em}{Sebastian H\"ohna}{
    Ludwig Maximilian University of Munich, Germany
}

\subsection*{Biodiversity research}

\textbf{Cambodia} \\
\cvitem{10em}{Thy Neang}{
    Wild Earth Allies, Phnom Penh, Cambodia
}

\cvitem{10em}{Lang Sokun}{
    Steung Hav Forestry Administration Unit of Prasihnuk Contonoment, Sihanoukville, Cambodia
}

\textbf{Malaysia} \\
\cvitem{10em}{Evan S.H.\ Quah}{
    Institute of Tropical Biodiversity and Sustainable Development, Universiti Malaysia Terengganu, Malaysia
}

\cvitem{10em}{M.S.\ Shahrul Anuar}{
    School of Biological Sciences, Universiti Sains Malaysia, Penang, Malaysia
}

\cvitem{10em}{Mohd Abdul Muin}{
    Centre for Global Sustainability Studies, Universiti Sains Malaysia, Penang, Malaysia
}

\textbf{Myanmar} \\
\cvitem{10em}{Myint Kyaw Thura}{
    Myanmar Environment Sustainable Conservation, Yangon, Myanmar
}

\cvitem{10em}{Aung Lin}{
    Fauna and Flora International, Sanchaung Township, Myanmar
}


\textbf{Thailand} \\
\cvitem{10em}{Anachelee Aowphol}{
    Department of Zoology, Kasetsart University, Bangkok, Thailand
}

\cvitem{10em}{Attapol Rujirawan}{
    Department of Zoology, Kasetsart University, Bangkok, Thailand
}

\textbf{Central Asia} \\
\cvitem{10em}{Natalia B.\ Ananjeva}{
    Zoological Institute, Russian Academy of Sciences, Russia
}

\textbf{Singapore} \\
\cvitem{10em}{Chan Kin Onn}{
    Lee of Kong Chian Natural History Museum, National University of Singapore
}


\end{document}
%%%%%%%%%%%%%%%%%%%%%%%%%%%%%%%%%%%%%%%%%%%%%%%%%%%%%%%%%%%%%%%%%%%%%%%%%%%%%%%
%%%%%%%%%%%%%%%%%%%%%%%%%%%%%%%%%%%%%%%%%%%%%%%%%%%%%%%%%%%%%%%%%%%%%%%%%%%%%%%
