While my approach to each subject varies, my primary philosophy remains the
same:
\textbf{\textit{My objective as an educator is to impart that biology is an
open-ended, inquiry-based endeavor that inherently requires critical thinking
and creativity.}}
% As with all fields of science, it is not a subject that can be effectively
% learned from a textbook alone.
I try to blur the distinction between teaching and research and incorporate
active, inquiry-based learning techniques and primary literature into my
teaching at all levels of undergraduate and graduate education.
My goal is not only to teach the subject material, but also expose students to
the venture of science, and hopefully inspire students to consider a
research-based career. 

% \subsection*{My Approach}
My general approach to teaching can be summarized in five components:
% (1) preparation,
% (2) establishing an interactive learning environment,
% (3) conveying concepts,
% (4) making the material relevant,
% and
% (5) measuring success.
\textbf{(1) Preparation}---I always strive to understand the concepts behind,
and the literature beyond, the course material.
% Without a thorough understanding of the subject, good teaching is impossible.
\textbf{(2) Establish learning environment}---I use enthusiasm, discussion
questions, and active-learning exercises to establish a comfortable,
interactive learning environment in the classroom 
% One effective strategy I have found is to be enthusiastic and engaging in my
% presentation of the material; this helps make the experience more enjoyable for
% the students and myself, and is enough to generate classroom participation in
% most cases.
% Using some appropriate humor can aid this process; I have found that making fun
% of myself in a lighthearted manner can be very effective for creating a
% comfortable environment for participation and help make me approachable to the
% students outside of class.
% When I get a group of students that are reluctant to participate, I kindly
% force participation and reward it with positive feedback.
% I do this by frequently asking questions and waiting for their feedback.
% Once the students realize their feedback is required to proceed and is never
% received negatively, even the shyest group of students will get involved.
\textbf{(3) Convey concepts}---My goal is to convey the important principles
and concepts behind the material, and prevent students from simply memorizing
information.
% I have two main strategies for helping students truly understand the material.
To accommodate the diversity of ways in which students learn effectively, I
often present concepts from several different perspectives.
In order to encourage creativity and synthesis of newly acquired knowledge,
I constantly ask questions and incorporate inquiry-based activities that
require the students to go beyond the class material and build upon the
concepts they are learning.
% These questions usually have many correct answers, and so at the same time,
% this strategy can help reinforce participation, because every contribution has
% value.
% The key is to wait for the students to work it out themselves; I will often
% rephrase the question or give hints to guide their path of thought, but I never
% answer the question for them.
% To augment these questions, I incorporate inquiry-based activities and
% information from the primary literature as often as possible. 
\textbf{(4) Make material relevant}---At every opportunity, I relate concepts
to current events, my research, or the research of other members of the
department.
This provides an opportunity for the students to learn about ongoing
research at their institution and how the material is important and relevant.
I also include current literature to convey to the students that topics in
biology are open-ended and constantly advancing.
% Most importantly, I incorporate hands-on activities to teach topics, because
% there is no better way to make a subject relevant than personal experience.
\textbf{(5) Measuring learning success}---I use conceptual inventories to
collect quantitative data on how well students are learning core concepts.
This allows me to test for general patterns of teaching effectiveness, and take
an evidence-based approach to my content, presentation, and exercises.
