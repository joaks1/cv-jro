As a tutor, teaching assistant, workshop
instructor, professor, and prison-education volunteer,
I have had the privilege of teaching a variety of biology subjects to a
diversity of learners.
While my approach to each subject and group of students varies, my primary
philosophy remains the same:
\textbf{\textit{My objective as an educator is to impart that
        science
        is an open-ended, inquiry-based endeavor that inherently requires
        critical thinking and creativity.}}
I try to blur the distinction between teaching and research and incorporate
active, inquiry-based learning techniques and primary literature into my
classes at all levels of
education.
My goal is not only to teach the subject material, but also expose students to
the venture of science, and hopefully inspire students to consider a
career in research.

% My general approach to teaching can be summarized in five components:
% \textbf{(1) Preparation}---I always strive to understand the concepts behind,
% and the literature beyond, the course material.
% \textbf{(2) Establish learning environment}---I use enthusiasm, discussion
% questions, and active-learning exercises to establish a comfortable,
% interactive learning environment in the classroom.
% \textbf{(3) Convey concepts}---My goal is to convey the important principles
% and concepts behind the material, and prevent students from simply memorizing
% information.
% To accommodate the diversity of ways in which students learn effectively, I
% often present concepts from several different perspectives.
% In order to encourage creativity and synthesis of newly acquired knowledge,
% I constantly ask questions and incorporate inquiry-based activities that
% require the students to go beyond the class material and build upon the
% concepts they are learning.
% \textbf{(4) Make material relevant}---At every opportunity, I relate concepts
% to current events, my research, or the research of other members of the
% department.
% This provides an opportunity for the students to learn about ongoing
% research at their institution and how the material is important and relevant.
% I also include current literature to convey to the students that topics in
% biology are open-ended and constantly advancing.
% \textbf{(5) Measure learning success}---I use conceptual inventories to
% collect quantitative data on how well students are learning core concepts.
% This allows me to test for general patterns of teaching effectiveness, and take
% an evidence-based approach to my content, presentation, and exercises.

%\paragraph*{Teaching experiences at AU}
\textit{\textbf{Teaching experiences at AU ---}}
During my time at AU,
I have developed and taught two graduate-level courses
and two predominantly undergraduate courses.
In addition to these courses, I have been involved in a number of efforts to
improve science education at AU and beyond.
%
To improve graduate student education at AU,
I am a member of the Informatics Steering Committee that developed a
curriculum for our students to earn a
\href{http://bulletin.auburn.edu/thegraduateschool/graduatedegreesoffered/biologicalsciencesmsphd_major/computationalbiology_gradcert/}{graduate certificate in computational biology}.
To assess and improve undergraduate education in the Department of Biological
Sciences, I have helped establish student learning outcomes (SLO) for our
majors, along with assessment tools for each outcome.
I was a member of the committee that developed the SLO for written
communication, and served as an ad hoc contributor to write and ``field test''
the assessment tools for the SLO for evolution.
%
For our annual
\href{http://www.auburn.edu/cosam/bioinformatics/}{Bioinformatics Bootcamp}
at AU,
I developed active-learning exercises,
including a module on using version control to improve reproducibility in
science that involves a
\href{http://phyletica.org/slides/git-intro/}{``hands-on'' lecture}
followed by a
\href{https://github.com/joaks1/au-bootcamp-git-intro}{group exercise}.
Due to these efforts, I was invited to participate in a symposium at the
iEvoBio 2019 Conference about how best to use computation for teaching biology,
which we have synthesized into a
white paper
(Wright et al.\ \citeyear{Wright2019}).
% \cite{Wright2019}.

% \paragraph*{Prison Education}
\textit{\textbf{Prison education ---}}
To bring science education to more diverse and underserved learners,
I have been working with
the Alabama Prison + Arts Education Project (APAEP) to develop and teach three
14-week courses in evolutionary biology to adult prisoners in correctional
facilities across Alabama.
The classes provided by APAEP have predominantly focused on the
arts and humanities, and
my
goal is to incorporate a broad set of STEM courses into the
curriculum.
I secured funding from the NSF
(\href{https://www.nsf.gov/awardsearch/showAward?AWD_ID=1656004&HistoricalAwards=false}{DEB 1656004})
to support these efforts.
% I am currently seeking funding in the outreach components of my
% NSF proposals
% to incorporate more computationally focused classes.
My students in Alabama correctional facilities are exceptionally hard-working
and eager to learn.
Working with them has been a privilege and, by far, my most rewarding
experience as a teacher.

% \paragraph*{Mentoring}
\textit{\textbf{Mentoring ---}}
Getting to work with a diverse group of talented undergraduate and graduate
students and postdocs in my lab is the most rewarding part of my job as a
PI.
I work with each of my lab members to develop an individual development
plan that fits
their level of experience, learning style, and career goals.
Most importantly, I try to create an atmosphere in my lab where
lab members work \emph{with} me, as opposed to \emph{for} me.

I am incredibly lucky to have recruited such an amazing group of students and
postdocs, who make
our lab group fun and productive.
While at AU, I have graduated two M.Sc.\ students, both of whom are in Ph.D.\
programs, and one was awarded an NSF Graduate Research Fellowship.
I currently have nine Ph.D. students in my lab, and serve on the advisory
committee of 10 other graduate students.
We have had 15 undergraduate student researchers in our lab, including four
NSF-funded REU students.
All three students in our first cohort of undergraduate researchers are now in
biology graduate programs.
I have mentored three postdocs.
The first is now an assistant professor at La Sierra University, the second is
a quantitative ecologist for the U.S.\ Geological Survey, and the third is
% still working in my lab, funded by an NSF award
% (\href{https://www.nsf.gov/awardsearch/showAward?AWD_ID=1656004&HistoricalAwards=false}{DEB 1656004}).
now a postdoc at the University of Michigan.
