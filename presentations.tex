\myHangIndent
{\bf Oaks, J.R.}
Generalizing phylogenetics to infer shared evolutionary events.
St.\ Cloud State University,
Department of Biology,
St.\ Cloud,
Minnesota, February 2016.
Invited Talk.

\myHangIndent
{\bf Oaks, J.R.}
Accommodating clustered divergences in phylogenetic inference.
University of Oklahoma,
Department of Biology,
Norman,
Oklahoma, October 2015.
Invited Talk.

\myHangIndent
{\bf Oaks, J.R.}
Improving inference of evolutionary history: Developing comparative methods for
genomic data.
Auburn University,
Department of Biological Sciences,
Auburn,
Alabama, March 2015.
Invited Talk.

\myHangIndent
{\bf Oaks, J.R.}
Improving inference of evolutionary history: Developing comparative methods for
genomic data.
Oklahoma State University Department of Integrative Biology, Stillwater,
Oklahoma, November 2014.
Invited Talk.

\myHangIndent
{\bf Oaks, J.R.}
An improved approximate-Bayesian method for estimating shared
evolutionary history.
Evolution 2014, Ernst Mayr Competition, Raleigh, North Carolina, June 2014.
Talk.

\myHangIndent
{\bf Oaks, J.R.}
An improved nonparametric approximate-Bayesian method for estimating shared
evolutionary history.
EVO-WIBO 2014: $6^{th}$ meeting of evolutionary biologists of the Pacific
Northwest, Port Townsend, Washington, April 2014.
Talk.

\myHangIndent
{\bf Oaks, J.R.}, J. Sukumaran, J.A. Esselstyn, C.W. Linkem, C.D. Siler, R.M.
Brown.
ABC, not as easy as 1, 2, 3: The potential perils of model choice via
approximate Bayesian computation.
Cornell University Department of Ecology and Evolutionary Biology, Ithaca,
New York, March 2013.
Invited talk.

\myHangIndent
{\bf Oaks, J.R.}, J. Sukumaran, J.A. Esselstyn, C.W. Linkem, C.D. Siler, R.M.
Brown.
ABC, not as easy as 1, 2, 3: The potential perils of model choice via
approximate Bayesian computation.
KU Ecology and Evolutionary Biology Department Graduate Student Organization
Retreat, University of Kansas, Lawrence, Kansas, November 2012.
Talk.

\myHangIndent
{\bf Oaks, J.R.}, J. Sukumaran, J.A. Esselstyn, C.W. Linkem, C.D. Siler, R.M.
Brown.
Comparative phylogeography of terrestrial vertebrates across Philippine
Pleistocene aggregate island complexes.
Evolution 2010, Portland, Oregon, June 2010.
Talk.

\myHangIndent
*{\bf Lusher, E.} and J.R. Oaks.
Phylogeography of Ringneck Snakes across Kansas.
University of Kansas Undergraduate Research Symposium, Lawrence, Kansas, April
2010.
Poster.
*Mentored undergraduate.

\myHangIndent
{\bf Oaks, J.R.}
Objective partition choice and the phylogenetic systematics and biogeography of
the true crocodiles (\emph{Crocodylus}).
KU Department of Ecology and Evolutionary Biology Seminar Series, Lawrence,
Kansas, February 2010.
Invited talk.

\myHangIndent
{\bf Oaks, J.R.}
Objective partition choice and the phylogenetic systematics and biogeography of
the true crocodiles.
Joint Meeting of Ichthyologists and Herpetologists, Portland, Oregon, July
2009.
Talk (Awarded best student paper in Systematics/Evolution).

\myHangIndent
{\bf Oaks, J.R.} and C.W. Linkem.
Accommodating Among-Site Rate Variation in Phylogenetic Inference: Data
Partitioning as a Random Variable and the Objective Choice of Partition
Strategy.
Evolution 2009, Ernst Mayr Competition, University of Idaho, Moscow, Idaho,
June 2009.
Talk.

\myHangIndent
{\bf Linkem, C.W.} and J.R. Oaks.
Examination and Utility of the Dirichlet Process Prior in Bayesian
Phylogenetics: A Test with Scincid and Anolis Lizards.
Evolution 2009, Ernst Mayr Competition, University of Idaho, Moscow, Idaho,
June 2009.
Talk.

\myHangIndent
{\bf Oaks, J.R.}
Approximate Bayesian Computation in the Chiricahua Mountains.
KU Natural History Museum Seminar Series, University of Kansas, Lawrence,
Kansas, May 2009.
Talk.

\myHangIndent
{\bf Oaks, J.R.}
Objective Partitioning in Phylogenetic Inference.
Sigma Xi Research Paper Competition, University of Kansas, Lawrence, Kansas,
April 2009.
Talk

\myHangIndent
{\bf Oaks, J.R.}
Accommodating Among-Site Rate Variation in Phylogenetic Inference: Data
Partitioning as a Random Variable and the Objective Choice of Partition
Strategy.
KU Natural History Museum Graduate Student Organization Retreat, University of
Kansas, Lawrence, Kansas, December 2008.
Talk.

\myHangIndent
{\bf Oaks, J.R.}
Evolution of the True Crocodiles (\emph{Crocodylus}).
Sigma Xi Research Paper Competition, University of Kansas, Lawrence, Kansas, April 2008.
Talk (Awarded 2nd place).

\myHangIndent
{\bf Oaks, J.R.}
Phylogenetic Systematics and Biogeography of the True Crocodiles
(\emph{Crocodylus}).
KU Natural History Museum Seminar Series, University of Kansas, Lawrence,
Kansas, November 2007.
Talk.

\myHangIndent
{\bf Carling, M.}, {\bf Z. Cheviron}, {\bf J. Grismer}, {\bf A. Jennings}, and
{\bf J.R. Oaks}.
Fieldwork at the LSU Museum of Natural Science: Where in the World are the
Museum Students?
Louisiana State University Museum of Natural Science Seminar Series, Baton
Rouge, Louisiana, April 2007.
Talk.

