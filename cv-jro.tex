\documentclass[10pt]{article}
\usepackage{anysize}
\marginsize{1in}{1in}{.5in}{.5in}
\pagenumbering{arabic}
\usepackage{setspace}
\usepackage[usenames]{color}
\usepackage[fleqn]{amsmath}
\usepackage{graphicx}
\usepackage{url}
\usepackage{verbatim}
\usepackage{indentfirst}
\usepackage{booktabs}
\usepackage{multirow}
\usepackage[table]{xcolor}
\usepackage{ragged2e}
\usepackage{xspace}
\usepackage{parskip}
\usepackage{tabulary}
\usepackage[normalem]{ulem}
\usepackage{hyperref}
\hypersetup{pdfborder={0 0 0}, colorlinks=true, urlcolor=blue, linkcolor=black}
\usepackage{titlesec}
\usepackage{lastpage}
\usepackage{fancyhdr}
\usepackage{ifthen}

\usepackage[round]{natbib}
\bibliographystyle{evolution}

%% Format headers and footers %%%%%%%%%%%%%%%%%%%%
\pagestyle{fancy}
%\lhead{\ifthenelse{\value{page}=1}{}{\sffamily\footnotesize Jamie Oaks}}
\lhead{\sffamily \emph{\docTitle} \\ Jamie R. Oaks}
%\chead{\ifthenelse{\value{page}=1}{{\scshape \docTitle} \\ Jamie Richard Oaks}{\sffamily\footnotesize \docTitle}}
%\rhead{\ifthenelse{\value{page}=1}{}{\sffamily\footnotesize \today}}
\rhead{\sffamily \today}
\cfoot{\sffamily\footnotesize Page \thepage\ of \pageref{LastPage}}
\renewcommand{\headrulewidth}{0.4pt}
\renewcommand{\footrulewidth}{0pt}

%% Format section titles %%%%%%%%%%%%%%%%%%%%%%%%%
\renewcommand\refname{Peer-reviewed Publications}

\titleformat{\section}[hang]
    {\large\sffamily\bfseries}
    {\S\ \thesection.}{.5em}{}[]
\titlespacing{\section}
    {0mm}{1.0ex plus .1ex minus .1ex}{-0.5ex}

\titleformat{\subsection}[hang]
    {\large\sffamily\itshape}
    {\S\ \thesection.}{.5em}{}[]
\titlespacing{\subsection}
    {0mm}{1.0ex plus .1ex minus .1ex}{-0.5ex}

\titleformat{\subsubsection}[runin]
    {\sffamily\bfseries}
    {\S\ \thesection.}{.5em}{}[.---]
\titlespacing{\subsubsection}
    {\parindent}{1.0ex plus .1ex minus .1ex}{0pt}

%% Format list environments %%%%%%%%%%%%%%%%%%%%%%%%
\renewcommand{\labelenumii}{\arabic{enumi}.\arabic{enumii}}
\renewcommand{\labelitemi}{$\circ$}

\newenvironment{myEnumerate}{
  \begin{enumerate}
    \setlength{\itemsep}{0.25em}
    \setlength{\parskip}{0pt}
    \setlength{\parsep}{0.5em}}
  {\end{enumerate}}

\newenvironment{myItemize}{
  \begin{itemize}
    \setlength{\leftskip}{-4mm}
    \setlength{\itemsep}{0.25em}
    \setlength{\parskip}{0pt}
    \setlength{\parsep}{0.5em}}
  {\end{itemize}}

%% Basic formatting and spacing %%%%%%%%%%%%%%%%%%%%%
\setlength{\parindent}{0em}
\setlength{\parskip}{0.5em}

%% My functions %%%%%%%%%%%%%%%%%%%%%%%%%%%%%
\newcommand{\ignore}[1]{}
\newcommand{\addTail}[1]{\textit{#1}.---}
\newcommand{\super}[1]{\ensuremath{^{\textrm{#1}}}}
\newcommand{\sub}[1]{\ensuremath{_{\textrm{#1}}}}
\newcommand{\dC}{\ensuremath{^\circ{\textrm{C}}}}
\newcommand{\tableSubItem}{\addtolength{\leftskip}{1em} \labelitemi \xspace}

%%%%%%%%%%%%%%%%%%%%%%%%%%%%%%%%%%%%%%%%%%%%%%%%%%%%%%%%%%%%%%
%%%%%%%%%%%%%%%%%%%%%%%%%%%%%%%%%%%%%%%%%%%%%%%%%%%%%%%%%%%%%%
\newcommand{\docTitle}{Curriculum Vitae\xspace}
\begin{document}
\raggedright
\singlespacing

\noindent\begin{tabular*}{\textwidth}[tb]{ @{}l @{\extracolsep{\fill}} l@{}}
%\noindent\begin{tabular}{ @{}p{4.25in} @{}p{2.75in}@{}}
Biodiversity Institute                                               & Phone: 785-864-3439 \\
Department of Ecology \& Evolutionary Biology     & Fax: 785-864-5335 \\
University of Kansas                                              & E-mail: \href{joaks1@gmail.com}{\tt joaks1@ku.edu} \\
1345 Jayhawk Boulevard                                       & \\
Lawrence, KS 66045                                              & \\
\end{tabular*}

\section*{Education}
%%%%%%%%%%%
\noindent\begin{tabulary}{\textwidth}{ @{} l L @{} }
%\noindent\begin{tabular}{ @{}p{1in}  >{\begin{minipage}[t]{5in}\raggedright\arraybackslash}l<{\end{minipage}\arraybackslash} }
2007--present	& Ph.D. in Ecology \& Evolutionary Biology, University of Kansas (anticipated May 2013) \\
			& \addtolength{\leftskip}{5mm}Advisors:  Rafe Brown \& Mark Holder \\[0.25em]
2004--2007	& M.S. in Biological Sciences, Louisiana State University \\[0.25em]
%			& \addtolength{\leftskip}{5mm}Thesis: Phylogenetic Systematics, Biogeography, and Evolutionary Ecology of the True Crocodiles (Eusuchia:  Crocodylidae:  \emph{Crocodylus}) \\
%			& Advisor:  Christopher Austin \\
1999--2004	& B.S. in Biology, University of Wisconsin Oshkosh (\emph{Summa Cum Laude}) \\
%			& \addtolength{\leftskip}{5mm}Emphasis:  Ecology and Organismal Biology \\
%			& \addtolength{\leftskip}{5mm}Minor:  Microbiology \\
%			& \addtolength{\leftskip}{5mm}Advisor:  Gregory Adler \\
%			& \addtolength{\leftskip}{5mm}Honors:  Summa Cum Laude \\
\end{tabulary}

\section*{Research/Curatorial Appointments}
%%%%%%%%%%%%%%%%%%%%%%%%
\noindent\begin{tabulary}{\textwidth}{ @{} l L @{} }
2010--2011	& Research Assistant, University of Kansas. \\
			& \tableSubItem Developed phylogenetic software with Dr. Mark Holder. \\[0.25em]
2007--2009	& Curatorial Assistant, University of Kansas Biodiversity Institute, Herpetology Collection. \\
			& \tableSubItem Performed full duties of Herpetology Collections Manager. \\[0.25em]
2008--2009	& Research Assistant, University of Kansas, NSF EPSCoR (EPS-0553722). \\
			& \tableSubItem Collected herpetological specimens, ecological data, and genetic samples across the Great Plains. \\[0.25em]
2006--2007	& Curatorial Assistant, Louisiana State University Museum of Natural Science, Herpetology Collection. \\
			& \tableSubItem Performed full duties of Herpetology Collections Manager. \\[0.25em]
2004--2006	& Research Assistant, Louisiana State University Museum of Natural Science. \\
			& \tableSubItem Geo-referenced specimen localities for NSF-funded HerpNET project. \\[0.25em]
2002--2004	& Undergraduate Research Assistant, University of Wisconsin Oshkosh. \\
			& \tableSubItem Analyzed long-term ecological data collected by Dr. Gregory Adler. \\[0.25em]
2002			& Undergraduate Research Assistant, University of Wisconsin Oshkosh. \\
			& \tableSubItem Conducted immunological study on alligators with Dr. Colleen McDermott. \\[0.25em]
2001			& Undergraduate Researcher, University of Wisconsin Oshkosh. \\
			& \tableSubItem Conducted population study of turtles and their parasites in Wisconsin. \\
\end{tabulary}

\section*{Teaching Appointments}
%%%%%%%%%%%%%%%%%%
\noindent\begin{tabulary}{\textwidth}{ @{} l L @{} }
2010		& Teaching Assistant, Introduction to Genetics (BIOL 350), University of Kansas (two semesters). \\
		& \tableSubItem Led weekly discussion sections (lectures and problem solving). \\
		& \tableSubItem Led exam review sessions. \\
		& \tableSubItem Received Award for Excellence in Teaching. \\[0.25em]
2010		& Teaching Assistant, Introduction to Biostatistics (BIOL 570), University of Kansas. \\
		& \tableSubItem Led weekly computer lab sections. \\[0.25em]
2010		& Guest Lecturer, Graduate-level Evolutionary Biology (BIOL 712), University of Kansas. \\
		& \tableSubItem Presented lecture on Statistical Phylogeography. \\[0.25em]
2006		& Teaching Assistant, Herpetology (BIOL 4146), Louisiana State University. \\
		& \tableSubItem Developed, organized, and presented weekly labs. \\[0.25em]
2003--2004	& Tutor for the University of Wisconsin Oshkosh Student Support Services \\
\end{tabulary}

%\nocite{*}
%\bibliography{jro}
\begin{thebibliography}{6}
%%%%%%%%%%%%%%
\providecommand{\natexlab}[1]{#1}
\providecommand{\url}[1]{\texttt{#1}}
\providecommand{\urlprefix}{URL }

\bibitem[{Oaks et~al.(Submitted)Oaks, Sukumaran, Esselstyn, Linkem, Siler,
  Holder, and Brown}]{Oaks2012}
Oaks, J.~R., J.~Sukumaran, J.~A. Esselstyn, C.~W. Linkem, C.~D. Siler, M.~T.
  Holder, and R.~M. Brown. Submitted to \emph{Evolution}.
\newblock Evidence for climate-driven diversification? a caution for
  interpreting {ABC} inferences of simultaneous historical events.

\bibitem[{Siler et~al.(In press)Siler, Oaks, Welton, Linkem, Swab, Diesmos, and
  Brown}]{Siler2012}
Siler, C.~D., J.~R. Oaks, L.~J. Welton, C.~W. Linkem, J.~Swab, A.~C. Diesmos,
  and R.~M. Brown. In press.
\newblock Did geckos ride the {P}alawan raft to the {P}hilippines?
\newblock \emph{Journal of Biogeography}.

\bibitem[{Oaks(2011)}]{Oaks2011}
Oaks, J.~R. 2011.
\newblock A time-calibrated species tree of {C}rocodylia reveals a recent
  radiation of the true crocodiles.
\newblock \emph{Evolution} 65:3285--3297.
\newblock
  \href{http://onlinelibrary.wiley.com/doi/10.1111/j.1558-5646.2011.01373.x/abstract}{link}.

\bibitem[{Siler et~al.(2010)Siler, Oaks, Esselstyn, Diesmos, and
  Brown}]{Siler2010}
Siler, C.~D., J.~R. Oaks, J.~A. Esselstyn, A.~C. Diesmos, and R.~M. Brown.
  2010.
\newblock Phylogeny and biogeography of {P}hilippine bent-toed geckos
  ({G}ekkonidae: \emph{{C}yrtodactylus}) contradict a prevailing model of
  {P}leistocene diversification.
\newblock \emph{Molecular Phylogenetics and Evolution} 55:699--710.
\newblock
  \href{http://www.sciencedirect.com/science/article/pii/S1055790310000382}{link}.

\bibitem[{Grismer et~al.(2008)Grismer, Neang, Chav, Perry L.~Wood, Oaks,
  Holden, Grismer, Szutz, and Youmans}]{GrismerL2008}
Grismer, L.~L., T.~Neang, T.~Chav, J.~Perry L.~Wood, J.~R. Oaks, J.~Holden,
  J.~L. Grismer, T.~R. Szutz, and T.~M. Youmans. 2008.
\newblock Additional amphibians and reptiles from the {P}hnom {S}amkos
  {W}ildlife {S}anctuary in {N}orthwestern {C}ardamom {M}ountains, {C}ambodia,
  with comments on their taxonomy and the discovery of four new species.
\newblock \emph{Raffles Bulletin of Zoology} 56:161--175.
\newblock
  \href{http://rmbr.nus.edu.sg/rbz/biblio/56/56rbz161-175.pdf}{link}.

\bibitem[{Oaks et~al.(2008)Oaks, Daul, and Adler}]{Oaks2008}
Oaks, J.~R., J.~M. Daul, and G.~H. Adler. 2008.
\newblock Life span of a tropical forest rodent, \emph{{P}roechimys
  semispinosus}.
\newblock \emph{Journal of Mammalogy} 89:904--908.
\newblock \href{http://dx.doi.org/10.1644/07-MAMM-A-045.1}{link}.

\end{thebibliography}

\section*{Presentations}
%%%%%%%%%%%%%
\hangindent=5mm
Oaks, J.R., J. Sukumaran, J.A. Esselstyn, C.W. Linkem, C.D. Siler, R.M. Brown.
Comparative phylogeography of terrestrial vertebrates across Philippine Pleistocene aggregate island complexes.
Evolution 2010, Portland, Oregon, June 2010.
Oral contribution.

\hangindent=5mm
*Lusher, E. and J.R. Oaks.
Phylogeography of Ringneck Snakes across Kansas.
University of Kansas Undergraduate Research Symposium, Lawrence, Kansas, April 2010.
Poster.
*Mentored undergraduate.

\hangindent=5mm
Oaks, J.R.
Objective partition choice and the phylogenetic systematics and biogeography of the true crocodiles (\emph{Crocodylus}).
KU Department of Ecology and Evolutionary Biology Seminar Series, Lawrence, Kansas, February 2010.
Invited oral contribution.

\hangindent=5mm
Oaks, J.R.
Objective partition choice and the phylogenetic systematics and biogeography of the true crocodiles.
Joint Meeting of Ichthyologists and Herpetologists, Portland, Oregon, July 2009.
Oral contribution (Awarded best student paper in Systematics/Evolution).

\hangindent=5mm
Oaks, J.R. and C.W. Linkem.
Accommodating Among-Site Rate Variation in Phylogenetic Inference: Data Partitioning as a Random Variable and the Objective Choice of Partition Strategy.
Evolution 2009, Ernst Mayr Competition, University of Idaho, Moscow, Idaho, June 2009.
Oral contribution.

\hangindent=5mm
Linkem, C.W. and J.R. Oaks.
Examination and Utility of the Dirichlet Process Prior in Bayesian Phylogenetics: A Test with Scincid and Anolis Lizards.
Evolution 2009, Ernst Mayr Competition, University of Idaho, Moscow, Idaho, June 2009.
Oral contribution.

\hangindent=5mm
Oaks, J.R.
Approximate Bayesian Computation in the Chiricahua Mountains.
KU Natural History Museum Seminar Series, University of Kansas, Lawrence, Kansas, May 2009.
Oral contribution.

\hangindent=5mm
Oaks, J.R.
Objective Partitioning in Phylogenetic Inference.
Sigma Xi Research Paper Competition, University of Kansas, Lawrence, Kansas, April 2009.
Oral contribution

\hangindent=5mm
Oaks, J.R.
Accommodating Among-Site Rate Variation in Phylogenetic Inference: Data Partitioning as a Random Variable and the Objective Choice of Partition Strategy.
KU Natural History Museum Graduate Student Organization Retreat, University of Kansas, Lawrence, Kansas, December 2008.
Oral contribution.

\hangindent=5mm
Oaks, J.R.
Evolution of the True Crocodiles (\emph{Crocodylus}).
Sigma Xi Research Paper Competition, University of Kansas, Lawrence, Kansas, April 2008.
Oral contribution (Awarded 2nd place).

\hangindent=5mm
Oaks, J.R.
Phylogenetic Systematics and Biogeograhy of the True Crocodiles (\emph{Crocodylus}).
KU Natural History Museum Seminar Series, University of Kansas, Lawrence, Kansas, November 2007.
Oral contribution.

\hangindent=5mm
Carling, M., Z. Cheviron, J. Grismer, A. Jennings, and J.R. Oaks.
Fieldwork at the LSU Museum of Natural Science: Where in the World are the Museum Students?
Louisiana State University Museum of Natural Science Seminar Series, Baton Rouge, Louisiana, April 2007.
Oral contribution.

\section*{Grants}
%%%%%%%%%
\hangindent=5mm
Oaks, J.R. (Co-PI), R. Brown (PI), and M. Holder (Co-PI).
Dissertation Research: Comparative Phylogeography of a Dynamic Archipelago.
NSF 1011423.
\$14,886.
08/01/2010--08/01/2012.
Funded.

\hangindent=5mm
Oaks, J.R.
Comparative Phylogeography of a Dynamic Archipelago.
KU Biodiversity Institute Panorama Grant.
\$1000.
01/05/2011--01/05/2012.
Funded.

\hangindent=5mm
Oaks, J.R.
Comparative Phylogeography of a Dynamic Archipelago.
KU Office of Graduate Studies Doctoral Student Research Award.
\$2000.
01/07/2011--01/07/2012.
Funded.

\ignore{
\hangindent=5mm
Oaks, J.R.
Comparative Phylogeography of a Dynamic Archipelago.
The Society for the Study of Evolution Rosemary Grant Graduate Research Award.
\$2,085.
01/07/2010--01/07/2011.
Not funded.
}
\hangindent=5mm
Oaks, J.R.
NSF stipend to attend a five-day short course, ``Comparative Methods and Macroevolution In R.''
Stipend covered travel and lodging; approximately \$1,000.
17--21/06/2010.
Funded.

\ignore{
\hangindent=5mm
Oaks, J.R.
Comparative Phylogeography of a Dynamic Archipelago.
KU Graduate Studies Summer Research Fellowship.
\$4,000.
05/2010--08/2010.
Not funded.
}
\hangindent=5mm
Oaks, J.R.
Comparative Phylogeography of a Dynamic Archipelago.
KU Department of Ecology and Evolutionary Biology Summer Fellowship.
\$4,000.
05/2010--08/2010.
Funded.

\hangindent=5mm
Oaks, J.R.
KU Department of Ecology and Evolutionary Biology Travel Grant.
\$300.
06/2010.
Funded.

\hangindent=5mm
Oaks, J.R.
KU Biodiversity Institute Travel Grant.
\$300.
06/2010.
Funded.

\hangindent=5mm
Oaks, J.R.
Comparative Phylogeography of Southern Indochina.
Society of Systematic Biologists Graduate Student Award.
\$1650.
06/09--06/10.
Funded.

\hangindent=5mm
Oaks, J.R.
Comparative Phylogeography of Southern Indochina.
Sigma Xi.
\$1000.
06/09--06/10.
Funded.

\ignore{
\hangindent=5mm
Oaks, J.R.
Comparative Phylogeography of Southern Indochina.
The Society for the Study of Amphibians and Reptiles Dean E. Metter Award.
\$800.
06/09--06/10.
Not funded.
}
\hangindent=5mm
Oaks, J.R.
KU Department of Ecology and Evolutionary Biology Travel Grant.
\$300.
05/2010.
Funded.

\hangindent=5mm
Oaks, J.R.
Estimating Species Trees Workshop Travel Grant.
University of Michigan Museum of Zoology.
\$500.
10--11/01/09.
Funded.

\hangindent=5mm
Oaks, J.R.
A Six-day Course in Statistical Phylogeography.
KU Biodiversity Institute Panorama Grant.
\$900.
6 days. 05/04/09--10/04/09.
Funded.

\hangindent=5mm
Oaks, J.R.
A Herpetofaunal Survey and Biogeographic Study of the Cardamom Mountains in Southeast Asia.
Louisiana State University BioGrads Grant.
\$300.
05/07--07/07.
Funded.

\hangindent=5mm
Oaks, J.R.
A Herpetofaunal Survey of Phnom Samkos, Cambodia.
Louisiana State University Museum of Natural Science Graduate Student Research Grant.
\$600.
05/07--07/07.
Funded.

\hangindent=5mm
Oaks, J.R.
The Biogeography of the Herpetofauna of the Cardamom Mountains.
Louisiana State University BioGrads Grant.
\$300.
05/06--07/06.
Funded.

\hangindent=5mm
Oaks, J.R.
A Herpetofaunal Survey of the Cardamom Mountains, Cambodia.
Louisiana State University Museum of Natural Science Graduate Student Research Grant.
\$300.
05/06--07/06.
Funded.

\hangindent=5mm
Oaks, J.R.
Herpetology Fieldwork in the Cardamom Mountains of Southeast Asia.
Louisiana State University BioGrads Grant.
\$300.
05/06--07/06.
Funded.

\hangindent=5mm
Oaks, J.R.
Geographic Variation of Multiple Paternity in the American alligator (\emph{Alligator mississippiensis}): Does Selection Rescue Genetic Diversity?
Sigma Xi.
\$1000.
01/05--12/05.
Funded.

\hangindent=5mm
Oaks, J.R.
Coevolution Among Turtles and Parasites: A Test of the Tenant.
University of Wisconsin Oshkosh Undergraduate Research Grant.
\$2500.
6/2001--6/2002.
Funded.

\section*{Awards \& Honors}
%%%%%%%%%%%%%%%
\hangindent=5mm
Oaks, J.R.
Kenneth B. Armitage Award for Excellence in Teaching.
BIOL 350---Introduction to Genetics.
\$250.
05/2011.

\hangindent=5mm
Oaks, J.R.
Henri Seibert award for best paper in Systematics/Evolution.
\$300.
Joint Meeting of Ichthyologists and Herpetologists, Portland, Oregon, 07/09.

\hangindent=5mm
Oaks, J.R.
Graduate Student Enhancement.
\$15,000.
Louisiana State University Department of Biological Sciences.
2004--2007.

\hangindent=5mm
Oaks, J.R.
Outstanding Biology Major Award.
\$500.
University of Wisconsin Oshkosh.
2004.

\hangindent=5mm
Oaks, J.R.
Leslie-Allen Fellowship in Ecology and Field Biology.
\$500.
University of Wisconsin Oshkosh.
2003.

\hangindent=5mm
Oaks, J.R.
Leslie-Allen Fellowship in Ecology and Field Biology.
\$500.
University of Wisconsin Oshkosh.
2002.

\section*{Professional Affiliations}
%%%%%%%%%%%%%%%%%
\begin{myItemize}
\item Society for the Study of Evolution
\item Society of Systematic Biologists
\item Society for the Study of Amphibians and Reptiles
\item American Society of Ichthyologists and Herpetologists
\end{myItemize}

\section*{Workshops}
%%%%%%%%%%%
\hangindent=5mm
Comparative Methods and Macroevolution in R (five-day course).
National Center for Ecological Analysis Synthesis, Santa Barbara, California.
17--21 June 2010.

\hangindent=5mm
Statistical Phylogeography Course.
American Museum of Natural History Southwestern Research Station, Arizona.
5--10 April 2009.

\hangindent=5mm
Estimating Species Trees: A Phylogenetic Paradigm for the 21st Century (Workshop).
Unveristy of Michigan, Ann Arbor, Michigan.
10--11 January 2009.

\hangindent=5mm
HerpNET Georeferencing Workshop.
University of Kansas, Lawrence, Kansas.
16--18 December 2004.

\section*{Service}
%%%%%%%%%
\begin{myItemize}
\item Ad hoc reviewer for \emph{Molecular Ecology} (2) and \emph{Molecular Biology and Evolution} (1)
\item Graduate student representative on the departmental committee for genomics faculty hire, University of Kansas Department of Ecology and Evolutionary Biology (Fall 2011)
\item Seminar Coordinator, Louisiana State University Museum of Natural Science Seminar Series (2006--2007)
\end{myItemize}

\section*{Outreach}
%%%%%%%%%%
\begin{myItemize}
\item While conducting fieldwork in Malaysia, I helped train undergraduate and graduate students from Universiti Kebangsaan Malaysia, Universiti Sains Malaysia, and La Sierra University in methods of field collection and specimen preparation (Fall 2011)
\item Trained two local high school student volunteers as fieldwork and curatorial assistants during Spring and Summer of 2008.
\item Louisiana State University Museum of Natural Science Children’s Special Saturday (2006)
	\begin{myItemize}
	\item Gave a presentation to children, ages 5--10, about the basic biology and diversity of frogs
	\end{myItemize}
\item Louisiana State University Museum of Natural Science Children’s Special Saturday (2005)
	\begin{myItemize}
	\item Introduced high school students to the imperative role of scientific research collections and curatorial methods.
	\end{myItemize}
\end{myItemize}

\end{document}
%%%%%%%%%%%%%%%%%%%%%%%%%%%%%%%%%%%%%%%%%%%%%%%%%%%%%%%%%%%%%%
%%%%%%%%%%%%%%%%%%%%%%%%%%%%%%%%%%%%%%%%%%%%%%%%%%%%%%%%%%%%%%
