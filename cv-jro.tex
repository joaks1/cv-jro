\documentclass[10pt]{article}
%\usepackage{anysize}
%\papersize{11in}{8.5in}
%\marginsize{1in}{1in}{.5in}{.5in}
\textwidth = 6.5 in
\textheight = 9 in
\oddsidemargin = 0.0 in
\evensidemargin = 0.0 in
\topmargin = -0.5 in
\headheight = 0.35 in
\headsep = 0.15 in
\topskip = 0 in
\footskip = 0.5 in
\pagenumbering{arabic}
\usepackage{setspace}
\usepackage[usenames]{color}
\usepackage[fleqn]{amsmath}
\usepackage{graphicx}
\usepackage{url}
\usepackage{verbatim}
\usepackage{indentfirst}
\usepackage{booktabs}
\usepackage{multirow}
\usepackage[table]{xcolor}
\usepackage{ragged2e}
\usepackage{xspace}
\usepackage{parskip}
\usepackage{tabulary}
\usepackage[normalem]{ulem}
\usepackage{hyperref}
\hypersetup{pdfborder={0 0 0}, colorlinks=true, urlcolor=blue, linkcolor=black}
\usepackage{titlesec}
\usepackage{lastpage}
\usepackage{fancyhdr}
\usepackage{ifthen}

\usepackage[round]{natbib}
\bibliographystyle{evolution}

%% Format headers and footers %%%%%%%%%%%%%%%%%%%%
\pagestyle{fancy}
%\lhead{\ifthenelse{\value{page}=1}{}{\sffamily\footnotesize Jamie Oaks}}
\lhead{\sffamily \emph{\docTitle} \\ Jamie R. Oaks}
%\chead{\ifthenelse{\value{page}=1}{{\scshape \docTitle} \\ Jamie Richard Oaks}{\sffamily\footnotesize \docTitle}}
%\rhead{\ifthenelse{\value{page}=1}{}{\sffamily\footnotesize \today}}
\rhead{\sffamily \today}
\cfoot{\sffamily\footnotesize Page \thepage\ of \pageref{LastPage}}
\renewcommand{\headrulewidth}{0.4pt}
\renewcommand{\footrulewidth}{0pt}

%% Format section titles %%%%%%%%%%%%%%%%%%%%%%%%%
\renewcommand\refname{Peer-reviewed Publications}

\titleformat{\section}[hang]
    {\large\sffamily\bfseries}
    {\S\ \thesection.}{.5em}{}[]
\titlespacing{\section}
    {0mm}{1.0ex plus .1ex minus .1ex}{-0.5ex}

\titleformat{\subsection}[hang]
    {\large\sffamily\itshape}
    {\S\ \thesection.}{.5em}{}[]
\titlespacing{\subsection}
    {0mm}{1.0ex plus .1ex minus .1ex}{-0.5ex}

\titleformat{\subsubsection}[runin]
    {\sffamily\bfseries}
    {\S\ \thesection.}{.5em}{}[.---]
\titlespacing{\subsubsection}
    {\parindent}{1.0ex plus .1ex minus .1ex}{0pt}

%% Format list environments %%%%%%%%%%%%%%%%%%%%%%%%
\renewcommand{\labelenumii}{\arabic{enumi}.\arabic{enumii}}
\renewcommand{\labelitemi}{$\circ$}

\newenvironment{myEnumerate}{
  \begin{enumerate}
    \setlength{\itemsep}{0.25em}
    \setlength{\parskip}{0pt}
    \setlength{\parsep}{0.5em}}
  {\end{enumerate}}

\newenvironment{myItemize}{
  \begin{itemize}
    \setlength{\leftskip}{-4mm}
    \setlength{\itemsep}{0.25em}
    \setlength{\parskip}{0pt}
    \setlength{\parsep}{0.5em}}
  {\end{itemize}}

%% Basic formatting and spacing %%%%%%%%%%%%%%%%%%%%%
\setlength{\parindent}{0em}
\setlength{\parskip}{0.5em}

%% My functions %%%%%%%%%%%%%%%%%%%%%%%%%%%%%
\newcommand{\ignore}[1]{}
\newcommand{\addTail}[1]{\textit{#1}.---}
\newcommand{\super}[1]{\ensuremath{^{\textrm{#1}}}}
\newcommand{\sub}[1]{\ensuremath{_{\textrm{#1}}}}
\newcommand{\dC}{\ensuremath{^\circ{\textrm{C}}}}
\newcommand{\tableSubItem}{\addtolength{\leftskip}{1em} \labelitemi \xspace}
\newcommand{\myHangIndent}{\hangindent=5mm}
\newcommand{\myIndent}{\hspace{5mm}}

%%%%%%%%%%%%%%%%%%%%%%%%%%%%%%%%%%%%%%%%%%%%%%%%%%%%%%%%%%%%%%
%%%%%%%%%%%%%%%%%%%%%%%%%%%%%%%%%%%%%%%%%%%%%%%%%%%%%%%%%%%%%%
\newcommand{\docTitle}{Curriculum Vitae\xspace}
\begin{document}
\raggedright
\singlespacing

\noindent\begin{tabular*}{\textwidth}[tb]{ @{}l @{\extracolsep{\fill}} l@{}}
%\noindent\begin{tabular}{ @{}p{4.25in} @{}p{2.75in}@{}}
Biodiversity Institute  & Phone: 785-864-3439 \\
Department of Ecology \& Evolutionary Biology & Fax: 785-864-5335 \\
University of Kansas    & E-mail: \href{mailto:joaks1@gmail.com}{\tt
                          joaks1@ku.edu} \\
1345 Jayhawk Boulevard  & \\
Lawrence, KS 66045      & \\
\end{tabular*}

\section*{Education}
%%%%%%%%%%%
\noindent\begin{tabulary}{\textwidth}{ @{} l L @{} }
%\noindent\begin{tabular}{ @{}p{1in}
    %>{\begin{minipage}[t]{5in}\raggedright\arraybackslash}l<{\end{minipage}\arraybackslash}
    %}
2007--present & Ph.D. in Ecology \& Evolutionary Biology, University of Kansas
                (anticipated May 2013) \\
              & \addtolength{\leftskip}{5mm}Advisors:  Rafe Brown \& Mark
                Holder \\[0.25em]
2004--2007    & M.S. in Biological Sciences, Louisiana State University
                \\[0.25em]
%             & \addtolength{\leftskip}{5mm}Thesis: Phylogenetic Systematics,
                %             Biogeography, and Evolutionary Ecology of the
                %             True Crocodiles (Eusuchia: Crocodylidae:
                %             \emph{Crocodylus}) \\
%             & Advisor:  Christopher Austin \\
1999--2004    & B.S. in Biology, University of Wisconsin Oshkosh (\emph{Summa
                Cum Laude}) \\
%             & \addtolength{\leftskip}{5mm}Emphasis:  Ecology and Organismal
                %             Biology \\
%             & \addtolength{\leftskip}{5mm}Minor:  Microbiology \\
%             & \addtolength{\leftskip}{5mm}Advisor:  Gregory Adler \\
%             & \addtolength{\leftskip}{5mm}Honors:  Summa Cum Laude \\
\end{tabulary}

\section*{Research/Curatorial Appointments}
%%%%%%%%%%%%%%%%%%%%%%%%
\noindent\begin{tabulary}{\textwidth}{ @{} l L @{} }
2010--present & Research Assistant, University of Kansas. \\
            & \tableSubItem Developed phylogenetic software with Dr.\ Mark
              Holder. \\[0.25em]
2007--2009  & Curatorial Assistant, University of Kansas Biodiversity
              Institute, Herpetology Collection. \\
            & \tableSubItem Performed full duties of Herpetology Collections
              Manager. \\[0.25em]
2008--2009  & Research Assistant, University of Kansas, NSF EPSCoR
              (EPS-0553722). \\
            & \tableSubItem Collected herpetological specimens, ecological
              data, and genetic samples across the Great Plains. \\[0.25em]
2006--2007  & Curatorial Assistant, Louisiana State University Museum of
              Natural Science, Herpetology Collection. \\
            & \tableSubItem Performed full duties of Herpetology Collections
              Manager. \\[0.25em]
2004--2006  & Research Assistant, Louisiana State University Museum of Natural
              Science. \\
            & \tableSubItem Geo-referenced specimen localities for NSF-funded
              HerpNET project. \\[0.25em]
2002--2004  & Undergraduate Research Assistant, University of Wisconsin
              Oshkosh. \\
            & \tableSubItem Analyzed long-term ecological data collected by Dr.
              Gregory Adler. \\[0.25em]
2002        & Undergraduate Research Assistant, University of Wisconsin
              Oshkosh. \\
            & \tableSubItem Conducted immunological study on alligators with
              Dr. Colleen McDermott. \\[0.25em]
2001        & Undergraduate Researcher, University of Wisconsin Oshkosh. \\
            & \tableSubItem Conducted population study of turtles and their
              parasites in Wisconsin. \\
\end{tabulary}

\section*{Teaching Appointments}
%%%%%%%%%%%%%%%%%%
\noindent\begin{tabulary}{\textwidth}{ @{} l L @{} }
2010    & Teaching Assistant, Introduction to Genetics (BIOL 350),
              University of Kansas (two semesters). \\
        & \tableSubItem Led weekly discussion sections (lectures and problem
              solving). \\
        & \tableSubItem Led exam review sessions. \\
        & \tableSubItem Received Award for Excellence in Teaching. \\[0.25em]
2010    & Guest Lecturer, Graduate-level Evolutionary Biology (BIOL 712),
              University of Kansas. \\
        & \tableSubItem Presented lecture on Statistical Phylogeography.
              \\[0.25em]
2009    & Teaching Assistant, Introduction to Biostatistics (BIOL 570),
              University of Kansas. \\
        & \tableSubItem Led weekly computer lab sections. \\[0.25em]
2006    & Teaching Assistant, Herpetology (BIOL 4146), Louisiana State
              University. \\
        & \tableSubItem Developed, organized, and presented weekly labs.
              \\[0.25em]
2003--2004 & Tutor for the University of Wisconsin Oshkosh Student Support
              Services \\
\end{tabulary}

%\nocite{*}
%\bibliography{jro}
\begin{thebibliography}{6}
%%%%%%%%%%%%%%
\providecommand{\natexlab}[1]{#1}
\providecommand{\url}[1]{\texttt{#1}}
\providecommand{\urlprefix}{URL }

\bibitem[{Oaks et~al.(Submitted)Oaks, Sukumaran, Esselstyn, Linkem, Siler,
  Holder, and Brown}]{Oaks2012}
{\bf Oaks, J.R.}, J.~Sukumaran, J.A. Esselstyn, C.W. Linkem, C.D. Siler, M.T.
  Holder, and R.M. Brown. In press.
\newblock Evidence for climate-driven diversification? A caution for
  interpreting {ABC} inferences of simultaneous historical events.
\newblock \emph{Evolution} doi:10.1111/j.1558-5646.2012.01840.x.
\newblock
  \href{http://onlinelibrary.wiley.com/doi/10.1111/j.1558-5646.2012.01840.x/abstract}{link}.

\bibitem[{Siler et~al.(In press)Siler, Oaks, Welton, Linkem, Swab, Diesmos, and
  Brown}]{Siler2012}
*Siler, C.D., {\bf J.R. Oaks}, L.J. Welton, C.W. Linkem, J. Swab, A.C. Diesmos,
  and R.M. Brown. 2012.
\newblock Did geckos ride the {P}alawan raft to the {P}hilippines?
\newblock \emph{Journal of Biogeography} 39:1217--1234.
\newblock
  \href{http://onlinelibrary.wiley.com/doi/10.1111/j.1365-2699.2011.02680.x/abstract}{link}.
  *Featured on journal cover.

\bibitem[{Oaks(2011)}]{Oaks2011}
*{\bf Oaks, J.R.} 2011.
\newblock A time-calibrated species tree of {C}rocodylia reveals a recent
  radiation of the true crocodiles.
\newblock \emph{Evolution} 65:3285--3297.
\newblock
  \href{http://onlinelibrary.wiley.com/doi/10.1111/j.1558-5646.2011.01373.x/abstract}{link}.
  *Featured in \emph{Nature}
  (\href{http://www.nature.com/nature/journal/v474/n7353/full/474545a.html}{link})
  and on journal cover.

\bibitem[{Siler et~al.(2010)Siler, Oaks, Esselstyn, Diesmos, and
  Brown}]{Siler2010}
Siler, C.D., {\bf J.R. Oaks}, J.A. Esselstyn, A.C. Diesmos, and R.M. Brown.
  2010.
\newblock Phylogeny and biogeography of {P}hilippine bent-toed geckos
  ({G}ekkonidae: \emph{{C}yrtodactylus}) contradict a prevailing model of
  {P}leistocene diversification.
\newblock \emph{Molecular Phylogenetics and Evolution} 55:699--710.
\newblock
  \href{http://www.sciencedirect.com/science/article/pii/S1055790310000382}{link}.

\bibitem[{Grismer et~al.(2008)Grismer, Neang, Chav, Perry L.~Wood, Oaks,
  Holden, Grismer, Szutz, and Youmans}]{GrismerL2008}
Grismer, L.L., T.~Neang, T.~Chav, J.~Perry L.~Wood, {\bf J.R. Oaks}, J.~Holden,
  J.L. Grismer, T.R. Szutz, and T.M. Youmans. 2008.
\newblock Additional amphibians and reptiles from the {P}hnom {S}amkos
  {W}ildlife {S}anctuary in {N}orthwestern {C}ardamom {M}ountains, {C}ambodia,
  with comments on their taxonomy and the discovery of four new species.
\newblock \emph{Raffles Bulletin of Zoology} 56:161--175.
\newblock
  \href{http://rmbr.nus.edu.sg/rbz/biblio/56/56rbz161-175.pdf}{link}.

\bibitem[{Oaks et~al.(2008)Oaks, Daul, and Adler}]{Oaks2008}
{\bf Oaks, J.R.}, J.M. Daul, and G.H. Adler. 2008.
\newblock Life span of a tropical forest rodent, \emph{{P}roechimys
  semispinosus}.
\newblock \emph{Journal of Mammalogy} 89:904--908.
\newblock \href{http://dx.doi.org/10.1644/07-MAMM-A-045.1}{link}.

\end{thebibliography}

\section*{Presentations (presenter in bold)}
%%%%%%%%%%%%%
\myHangIndent
{\bf Oaks, J.R.}, J. Sukumaran, J.A. Esselstyn, C.W. Linkem, C.D. Siler, R.M.
Brown.
ABC, not as easy as 1, 2, 3: The potential perils of model choice via
approximate Bayesian computation.
KU Ecology and Evolutionary Biology Department Graduate Student Organization
Retreat, University of Kansas, Lawrence, Kansas, November 2012.
Talk.

\myHangIndent
{\bf Oaks, J.R.}, J. Sukumaran, J.A. Esselstyn, C.W. Linkem, C.D. Siler, R.M.
Brown.
Comparative phylogeography of terrestrial vertebrates across Philippine
Pleistocene aggregate island complexes.
Evolution 2010, Portland, Oregon, June 2010.
Talk.

\myHangIndent
*{\bf Lusher, E.} and J.R. Oaks.
Phylogeography of Ringneck Snakes across Kansas.
University of Kansas Undergraduate Research Symposium, Lawrence, Kansas, April
2010.
Poster.
*Mentored undergraduate.

\myHangIndent
{\bf Oaks, J.R.}
Objective partition choice and the phylogenetic systematics and biogeography of
the true crocodiles (\emph{Crocodylus}).
KU Department of Ecology and Evolutionary Biology Seminar Series, Lawrence,
Kansas, February 2010.
Invited talk.

\myHangIndent
{\bf Oaks, J.R.}
Objective partition choice and the phylogenetic systematics and biogeography of
the true crocodiles.
Joint Meeting of Ichthyologists and Herpetologists, Portland, Oregon, July
2009.
Talk (Awarded best student paper in Systematics/Evolution).

\myHangIndent
{\bf Oaks, J.R.} and C.W. Linkem.
Accommodating Among-Site Rate Variation in Phylogenetic Inference: Data
Partitioning as a Random Variable and the Objective Choice of Partition
Strategy.
Evolution 2009, Ernst Mayr Competition, University of Idaho, Moscow, Idaho,
June 2009.
Talk.

\myHangIndent
{\bf Linkem, C.W.} and J.R. Oaks.
Examination and Utility of the Dirichlet Process Prior in Bayesian
Phylogenetics: A Test with Scincid and Anolis Lizards.
Evolution 2009, Ernst Mayr Competition, University of Idaho, Moscow, Idaho,
June 2009.
Talk.

\myHangIndent
{\bf Oaks, J.R.}
Approximate Bayesian Computation in the Chiricahua Mountains.
KU Natural History Museum Seminar Series, University of Kansas, Lawrence,
Kansas, May 2009.
Talk.

\myHangIndent
{\bf Oaks, J.R.}
Objective Partitioning in Phylogenetic Inference.
Sigma Xi Research Paper Competition, University of Kansas, Lawrence, Kansas,
April 2009.
Talk

\myHangIndent
{\bf Oaks, J.R.}
Accommodating Among-Site Rate Variation in Phylogenetic Inference: Data
Partitioning as a Random Variable and the Objective Choice of Partition
Strategy.
KU Natural History Museum Graduate Student Organization Retreat, University of
Kansas, Lawrence, Kansas, December 2008.
Talk.

\myHangIndent
{\bf Oaks, J.R.}
Evolution of the True Crocodiles (\emph{Crocodylus}).
Sigma Xi Research Paper Competition, University of Kansas, Lawrence, Kansas, April 2008.
Talk (Awarded 2nd place).

\myHangIndent
{\bf Oaks, J.R.}
Phylogenetic Systematics and Biogeography of the True Crocodiles
(\emph{Crocodylus}).
KU Natural History Museum Seminar Series, University of Kansas, Lawrence,
Kansas, November 2007.
Talk.

\myHangIndent
{\bf Carling, M.}, {\bf Z. Cheviron}, {\bf J. Grismer}, {\bf A. Jennings}, and
{\bf J.R. Oaks}.
Fieldwork at the LSU Museum of Natural Science: Where in the World are the
Museum Students?
Louisiana State University Museum of Natural Science Seminar Series, Baton
Rouge, Louisiana, April 2007.
Talk.

\section*{Grants}
%%%%%%%%%
\myHangIndent
Oaks, J.R. (Co-PI), R. Brown (PI), and M. Holder (Co-PI).
Comparative Phylogeography of a Dynamic Archipelago.
National Science Foundation Doctoral Dissertation Improvement Grant.
\$14,886.
8/2010--8/2012.
Funded.

\myHangIndent
Oaks, J.R.
Comparative Phylogeography of a Dynamic Archipelago.
KU Biodiversity Institute Panorama Grant.
\$1000.
5/2011--5/2012.
Funded.

\myHangIndent
Oaks, J.R.
Comparative Phylogeography of a Dynamic Archipelago.
KU Office of Graduate Studies Doctoral Student Research Award.
\$2000.
7/2011--7/2012.
Funded.

\ignore{
\myHangIndent
Oaks, J.R.
Comparative Phylogeography of a Dynamic Archipelago.
The Society for the Study of Evolution Rosemary Grant Graduate Research Award.
\$2,085.
7/2010--7/2011.
Not funded.
}
\myHangIndent
Oaks, J.R.
NSF stipend to attend a five-day short course, ``Comparative Methods and
Macroevolution In R.''
Stipend covered travel and lodging; approximately \$1,000.
6/2010.
Funded.

\ignore{
\myHangIndent
Oaks, J.R.
Comparative Phylogeography of a Dynamic Archipelago.
KU Graduate Studies Summer Research Fellowship.
\$4,000.
5/2010--8/2010.
Not funded.
}
\myHangIndent
Oaks, J.R.
Comparative Phylogeography of a Dynamic Archipelago.
KU Department of Ecology and Evolutionary Biology Summer Fellowship.
\$4,000.
5/2010--8/2010.
Funded.

\myHangIndent
Oaks, J.R.
KU Department of Ecology and Evolutionary Biology Travel Grant.
\$300.
6/2010.
Funded.

\myHangIndent
Oaks, J.R.
KU Biodiversity Institute Travel Grant.
\$300.
6/2010.
Funded.

\myHangIndent
Oaks, J.R.
Comparative Phylogeography of Southern Indochina.
Society of Systematic Biologists Graduate Student Award.
\$1650.
6/2009--6/2010.
Funded.

\myHangIndent
Oaks, J.R.
Comparative Phylogeography of Southern Indochina.
Sigma Xi.
\$1000.
6/2009--6/2010.
Funded.

\ignore{
\myHangIndent
Oaks, J.R.
Comparative Phylogeography of Southern Indochina.
The Society for the Study of Amphibians and Reptiles Dean E. Metter Award.
\$800.
6/2009--6/2010.
Not funded.
}
\myHangIndent
Oaks, J.R.
KU Department of Ecology and Evolutionary Biology Travel Grant.
\$300.
5/2010.
Funded.

\myHangIndent
Oaks, J.R.
Estimating Species Trees Workshop Travel Grant.
University of Michigan Museum of Zoology.
\$500.
1/2009.
Funded.

\myHangIndent
Oaks, J.R.
A Six-day Course in Statistical Phylogeography.
KU Biodiversity Institute Panorama Grant.
\$900.
4/2009.
Funded.

\myHangIndent
Oaks, J.R.
A Herpetofaunal Survey and Biogeographic Study of the Cardamom Mountains in
Southeast Asia.
Louisiana State University BioGrads Grant.
\$300.
5/2007--7/2007.
Funded.

\myHangIndent
Oaks, J.R.
A Herpetofaunal Survey of Phnom Samkos, Cambodia.
Louisiana State University Museum of Natural Science Graduate Student Research
Grant.
\$600.
5/2007--7/2007.
Funded.

\myHangIndent
Oaks, J.R.
The Biogeography of the Herpetofauna of the Cardamom Mountains.
Louisiana State University BioGrads Grant.
\$300.
5/2006--7/2006.
Funded.

\myHangIndent
Oaks, J.R.
A Herpetofaunal Survey of the Cardamom Mountains, Cambodia.
Louisiana State University Museum of Natural Science Graduate Student Research
Grant.
\$300.
5/2006--7/2006.
Funded.

\myHangIndent
Oaks, J.R.
Herpetology Fieldwork in the Cardamom Mountains of Southeast Asia.
Louisiana State University BioGrads Grant.
\$300.
5/2006--7/2006.
Funded.

\myHangIndent
Oaks, J.R.
Geographic Variation of Multiple Paternity in the American alligator
(\emph{Alligator mississippiensis}): Does Selection Rescue Genetic Diversity?
Sigma Xi.
\$1000.
1/2005--12/2005.
Funded.

\myHangIndent
Oaks, J.R.
Coevolution Among Turtles and Parasites: A Test of the Tenant.
University of Wisconsin Oshkosh Undergraduate Research Grant.
\$2500.
6/2001--6/2002.
Funded.

\section*{Awards \& Honors}
%%%%%%%%%%%%%%%
\myHangIndent
Oaks, J.R.
Kenneth B. Armitage Award for Excellence in Teaching.
BIOL 350---Introduction to Genetics.
\$250.
5/2011.

\myHangIndent
Oaks, J.R.
Henri Seibert award for best paper in Systematics/Evolution.
\$300.
Joint Meeting of Ichthyologists and Herpetologists, Portland, Oregon, 07/09.

\myHangIndent
Oaks, J.R.
Graduate Student Enhancement.
\$15,000 (3 years).
Louisiana State University Department of Biological Sciences.
2004--2007.

\myHangIndent
Oaks, J.R.
Outstanding Biology Major Award.
\$500.
University of Wisconsin Oshkosh.
2004.

\myHangIndent
Oaks, J.R.
Leslie-Allen Fellowship in Ecology and Field Biology.
\$500.
University of Wisconsin Oshkosh.
2003.

\myHangIndent
Oaks, J.R.
Leslie-Allen Fellowship in Ecology and Field Biology.
\$500.
University of Wisconsin Oshkosh.
2002.

\section*{Software}
\myHangIndent
{SAT}\'{e}: Phylogenetic software for simultaneous alignment and tree
estimation.
\href{http://phylo.bio.ku.edu/software/sate/sate.html}
{\tt http://phylo.bio.ku.edu/software/sate/sate.html}.
Authors: J. Yu, M.T. Holder, J. Sukumaran, S. Mirarab, and J. Oaks.

\myHangIndent
SeqSift: Package for manipulating molecular sequence data.
\href{https://github.com/joaks1/SeqSift}{\tt https://github.com/joaks1/SeqSift}.
Author: J. Oaks.

\section*{Professional Affiliations}
%%%%%%%%%%%%%%%%%
\begin{myItemize}
\item American Society of Ichthyologists and Herpetologists
\item International Biogeography Society
\item Society for the Study of Amphibians and Reptiles
\item Society for the Study of Evolution
\item Society of Systematic Biologists
\end{myItemize}

\section*{Workshops}
%%%%%%%%%%%
\myHangIndent
Comparative Methods and Macroevolution in R (five-day course).
National Center for Ecological Analysis Synthesis, Santa Barbara, California.
17--21 June 2010.

\myHangIndent
Statistical Phylogeography Course.
American Museum of Natural History Southwestern Research Station, Arizona.
5--10 April 2009.

\myHangIndent
Estimating Species Trees: A Phylogenetic Paradigm for the 21st Century.
Unversity of Michigan, Ann Arbor, Michigan.
10--11 January 2009.

\myHangIndent
HerpNET Georeferencing Workshop.
University of Kansas, Lawrence, Kansas.
16--18 December 2004.

\section*{Service}
%%%%%%%%%
\begin{myItemize}
\item Ad hoc reviewer for
    \emph{BMC Evolutionary Biology} (2),
    \emph{Evolution} (1),
    \emph{Molecular Biology and Evolution} (1),
    \emph{Molecular Ecology} (2),
    \emph{Molecular Phylogenetics and Evolution} (1), and
    \emph{Organisms Diversity and Evolution} (1)
\item Graduate student representative on the departmental committee for
    genomics faculty hire, University of Kansas Department of Ecology and
    Evolutionary Biology (Fall 2011)
\item Seminar Coordinator, Louisiana State University Museum of Natural Science
    Seminar Series (2006--2007)
\end{myItemize}

\section*{Outreach}
%%%%%%%%%%
\begin{myItemize}
\item While conducting fieldwork in Malaysia, I helped train undergraduate and
    graduate students from Universiti Kebangsaan Malaysia, Universiti Sains
    Malaysia, and La Sierra University in methods of field collection and
    specimen preparation (Fall 2011)
\item Trained two local high school student volunteers as fieldwork and
    curatorial assistants for the University of Kansas NSF EPSCoR project
    (2008).
\item Louisiana State University Museum of Natural Science Children's Special
    Saturday (2006)
    \begin{myItemize}
        \item Gave a presentation to children, ages 5--10, about the basic
            biology and diversity of frogs
    \end{myItemize}
\item Louisiana State University Museum of Natural Science Children's Special
    Saturday (2005)
    \begin{myItemize}
        \item Introduced high school students to the imperative role of
            scientific research collections and curatorial methods.
    \end{myItemize}
\end{myItemize}

\section*{Additional Teaching Experience}
\myHangIndent
Teaching Assistant for Workshop on Advances in Multiple Sequence Alignment and
Phylogeny Estimation.  Smithsonian Institution, Washington, DC.
20--21 May 2012. \\
% \myIndent Helped develop software tutorials and assisted workshop participants.
\begin{itemize}
    \setlength{\leftskip}{0em}
    \setlength{\itemsep}{0.25em}
    \setlength{\parskip}{-1em}
    \setlength{\parsep}{0.5em}
    \item Helped write tutorials and assisted workshop participants with
        phylogenetic analyses.
\end{itemize}

\section*{Additional Research Experience \& Skills}
%%%%%%%%%%%%%%%%%%%%%%%%%%%
\myHangIndent
{\sffamily\itshape Computer programming languages:} Python, R, Java, PERL,
shell scripting and \LaTeX; minimal experience with C/C++.

\myHangIndent
{\sffamily\itshape Fieldwork:} Cambodia (2012, 2006); Malaysia (2011, 2006);
Great Plains, USA (2007-2009); Southeastern USA (2004-2007); Florida, USA
(2002); Wisconsin, USA (2001).

% \newpage
\section*{References}
% %%%%%%%%%%%%%%%%%%%%%%%%%%%
Rafe Brown \\
\myIndent Curator of Herpetology, Biodiversity Institute \\
\myIndent Associate Professor, Department of Ecology \& Evolutionary Biology \\
\myIndent University of Kansas \\
\myIndent Dyche Hall, 1345 Jayhawk Boulevard \\
\myIndent Lawrence, KS 66045 \\
\myIndent (785) 864-3403 \\
\myIndent \href{mailto:rafe@ku.edu}{\tt rafe@ku.edu}

Scott Freeman \\
\myIndent Principal Lecturer, Department of Biology \\
\myIndent University of Washington \\
\myIndent Box 351800 \\
\myIndent Seattle, WA 98195\\
\myIndent (206) 300-4448 \\
\myIndent \href{mailto:srf991@u.washington.edu}{\tt srf991@u.washington.edu}

Mark Holder \\
\myIndent Associate Professor, Department of Ecology \& Evolutionary Biology \\
\myIndent University of Kansas \\
\myIndent 1200 Sunnyside Avenue \\
\myIndent Lawrence, KS 66045 \\
\myIndent (785) 856-2873 \\
\myIndent \href{mailto:mtholder@ku.edu}{\tt mtholder@ku.edu}

% Lee Grismer \\
% \myIndent Professor, Department of Biology \\
% \myIndent La Sierra University \\
% \myIndent 4500 Riverwalk Parkway \\
% \myIndent Riverside, CA 92515 \\
% \myIndent (951) 785-2345 \\
% \myIndent \href{mailto:lgrismer@lasierra.edu}{\tt lgrismer@lasierra.edu}

% Adam Leach\'{e} \\
% \myIndent Assistant Professor, Department of Biology \\
% \myIndent University of Washington \\
% \myIndent Box 351800 \\
% \myIndent Seattle, WA 98195\\
% \myIndent (206) 543-7622 \\
% \myIndent \href{mailto:leache@uw.edu}{\tt leache@uw.edu}

% Vladimir Minin \\
% \myIndent Associate Professor, Departments of Statistics and Biology\\
% \myIndent University of Washington \\
% \myIndent Box 354322 \\
% \myIndent Seattle, WA 98195\\
% \myIndent (206) 543-4302 \\
% \myIndent \href{mailto:vminin@uw.edu}{\tt vminin@uw.edu}

% Cameron Siler \\
% \myIndent Assistant Curator of Herpetology, Sam Noble Museum \\
% \myIndent Assistant Professor, Department of Biology \\
% \myIndent University of Oklahoma \\
% \myIndent 2401 Chautauqua Avenue \\
% \myIndent Norman, OK 73072\\
% \myIndent (405) 325-3718 \\
% \myIndent \href{mailto:camsiler@ou.edu}{\tt camsiler@ou.edu}



\end{document}
%%%%%%%%%%%%%%%%%%%%%%%%%%%%%%%%%%%%%%%%%%%%%%%%%%%%%%%%%%%%%%
%%%%%%%%%%%%%%%%%%%%%%%%%%%%%%%%%%%%%%%%%%%%%%%%%%%%%%%%%%%%%%
