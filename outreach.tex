\cvitem{6em}{2019}{Prison education---Alabama Prison Arts + Education Project (APAEP)
    \begin{itemize}
        \item Teaching evolutionary biology course to a diverse class of 20
            students incarcerated at the Staton Correctional Facility.
            \url{https://github.com/phyletica/apaep-evolution}
    \end{itemize}
}

% \cvitem{6em}{2019}{Prison education---Alabama Prison Arts + Education Project (APAEP)
%     \begin{itemize}
%         \item Worked with Wayne Barger to teach evolutionary botany an Alabama
%             Correctional Facility.
%     \end{itemize}
% }

\cvitem{6em}{2019}{Alabama State Science Olympiad
    \begin{itemize}
        \item Developed and administered practical exam in herpetology for the
            middle and high school competitions.
    \end{itemize}
}

\cvitem{6em}{2018}{Prison education---Alabama Prison Arts + Education Project (APAEP)
    \begin{itemize}
        \item Developed and taught an evolutionary biology course to a diverse
            class of 39 students incarcerated at the Ventress Correctional Facility
            in Clayton, Alabama.
            \url{https://github.com/phyletica/apaep-evolution}
    \end{itemize}
}

\cvitem{6em}{2018}{Alabama State Science Olympiad
    % \begin{itemize}
    %     \item Developed and administered practical exam in herpetology for the
    %         middle and high school competitions.
    % \end{itemize}
}

\cvitem{6em}{2018}{Open House of the Auburn University Museum of Natural History
    \begin{itemize}
        \item Led tours through the AUMNH Research Collections for
            members of the public of all ages
    \end{itemize}
}

\cvitem{6em}{2017}{Open House of the Auburn University Museum of Natural History
    % \begin{itemize}
    %     \item Led tours through the AUMNH Amphibian and Reptiles Collections for
    %         members of the public of all ages
    % \end{itemize}
}

\cvitem{6em}{2016}{Open House of the Auburn University Museum of Natural History
    % \begin{itemize}
    %     \item Led tours through the AUMNH Amphibian and Reptiles Collections for
    %         members of the public of all ages
    % \end{itemize}
}

\cvitem{6em}{2016}{Project Lead the Way (PLTW) Alabama State Conference
    \begin{itemize}
        \item Gave a presentation about bioinformatics to K-12 educators from
            across Alabama
    \end{itemize}
}

\cvitem{6em}{2013--present}{
    Blog contributor:
    \href{http://phyletica.org/posts/}{phyletica.org}
}

\cvitem{6em}{2014}{
Contributed a short story about my research to
\href{https://sciworthy.com}{Sciworthy.com}, a news site with the goal of
improving public understanding of science.
\url{https://sciworthy.com/whos-the-living-fossil/}
}

\longcontent{
\cvitem{6em}{2011}{
While conducting fieldwork in Malaysia, I helped train undergraduate and
graduate students from Universiti Kebangsaan Malaysia, Universiti Sains
Malaysia, and La Sierra University in methods of field collection and specimen
preparation.
}

\cvitem{6em}{2008}{
Trained two local high school student volunteers as fieldwork and curatorial
assistants for the University of Kansas NSF EPSCoR project.
}

\cvitem{6em}{2006}{
Louisiana State University Museum of Natural Science Children's Special
Saturday
\begin{itemize}
    \item Gave a presentation to children, ages 5--10, about the basic biology
        and diversity of frogs
\end{itemize}
}

\cvitem{6em}{2005}{
Louisiana State University Museum of Natural Science Children's Special
Saturday
\begin{itemize}
    \item Introduced high school students to the imperative role of scientific
        research collections and curatorial methods
\end{itemize}
}
}
